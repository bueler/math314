\documentclass[11pt]{amsart}
%\pagestyle{empty} 
\setlength{\topmargin}{-0.5in} % usually -0.25in
\addtolength{\textheight}{1.2in} % usually 1.25in
\addtolength{\oddsidemargin}{-0.95in}
\addtolength{\evensidemargin}{-0.95in}
\addtolength{\textwidth}{1.9in} %\setlength{\parindent}{0pt}

\newcommand{\normalspacing}{\renewcommand{\baselinestretch}{1.1}\tiny\normalsize}
\normalspacing

% macros
\usepackage{amssymb,xspace,alltt,verbatim}
\usepackage[final]{graphicx}
\usepackage[pdftex,colorlinks=true]{hyperref}
\usepackage{fancyvrb}
\usepackage{tikz}

\newtheorem*{lem*}{Lemma}

\newcommand{\ba}{\mathbf{a}}
\newcommand{\bb}{\mathbf{b}}
\newcommand{\bc}{\mathbf{c}}
\newcommand{\bd}{\mathbf{d}}
\newcommand{\bs}{\mathbf{s}}
\newcommand{\bu}{\mathbf{u}}
\newcommand{\bv}{\mathbf{v}}
\newcommand{\bw}{\mathbf{w}}
\newcommand{\bx}{\mathbf{x}}
\newcommand{\by}{\mathbf{y}}

\newcommand{\bbf}{\mathbf{f}}

\newcommand{\CC}{{\mathbb{C}}}
\newcommand{\RR}{{\mathbb{R}}}
\newcommand{\eps}{\epsilon}
\newcommand{\ZZ}{{\mathbb{Z}}}
\newcommand{\ZZn}{{\mathbb{Z}}_n}
\newcommand{\NN}{{\mathbb{N}}}
\newcommand{\ip}[2]{\mathrm{\left<#1,#2\right>}}

\renewcommand{\Re}{\operatorname{Re}}
\renewcommand{\Im}{\operatorname{Im}}

\newcommand{\Log}{\operatorname{Log}}

\newcommand{\Span}{\operatorname{span}}

\newcommand{\grad}{\nabla}

\newcommand{\ds}{\displaystyle}

\newcommand{\Matlab}{\textsc{Matlab}\xspace}
\newcommand{\Octave}{\textsc{Octave}\xspace}
\newcommand{\pylab}{\textsc{pylab}\xspace}

\newcommand{\prob}[1]{\bigskip\noindent\textbf{#1.} }
\newcommand{\pts}[1]{(\emph{#1 pts})}

\newcommand{\probpts}[2]{\prob{#1} \pts{#2} \quad}
\newcommand{\ppartpts}[2]{\textbf{(#1)} \pts{#2} \quad}
\newcommand{\epartpts}[2]{\medskip\noindent \textbf{(#1)} \pts{#2} \quad}

\newcommand{\mybox}{\boxed{\phantom{\begin{bmatrix} I & I \\ O & O \end{bmatrix} fj ladsfj}}}
\newcommand{\dbox}{\boxed{\phantom{\begin{bmatrix} I & I \\ O & O \end{bmatrix}}}}


\begin{document}
\hfill \Large Name:\underline{\phantom{Ed Bueler really really long long long name}}
\medskip

\scriptsize \noindent Math 314 Linear Algebra (Bueler) \hfill Monday, 28 March 2022
\medskip

\LARGE\centerline{\textbf{Midterm Exam 2}}

\smallskip
\begin{quote}
\large
\textbf{No book, notes, electronics, calculator, or internet access.  100 points possible. 65 minutes maximum.}
\end{quote}

\normalsize
\medskip

\thispagestyle{empty}

\prob{1}  Here is a matrix and its row-reduced echelon form:

% R = [1 2 0 -1 1; 0 0 1 0 1; 0 0 0 0 0]
% C = reshape(-2:6,3,3)
% A = C*R
% A =
%        -2        -4         1         2        -1
%        -1        -2         2         1         1
%         0         0         3         0         3

    $$A = \begin{bmatrix} -2 & -4 & 1 & 2 & -1 \\ -1 & -2 & 2 & 1 & -1 \\ 0 & 0 & 3 & 0 & 3 \end{bmatrix}
      \qquad \to \qquad
      R = \begin{bmatrix} 1 & 2 & 0 & -1 & 1 \\ 0 & 0 & 1 & 0 & 1 \\ 0 & 0 & 0 & 0 & 0 \end{bmatrix}$$

\epartpts{a}{5}  What is the \textbf{dimension} of the row space $C(A^\top)$?  \textbf{Provide a basis} for $C(A^\top)$.  (\emph{Suggestion:}  Write your basis as $C(A^\top)=\Span\{\dots\}$ with particular vectors.)

\medskip
\noindent $\dim C(A^\top) = $ \dbox
\vfill

\epartpts{b}{5}  What is the \textbf{dimension} of the column space $C(A)$?  \textbf{Provide a basis}.

\medskip
\noindent $\dim C(A) = $ \dbox
\vfill

\epartpts{c}{5}  What is the \textbf{dimension} of the null space $N(A)$?  \textbf{Provide a basis}.

\medskip
\noindent $\dim N(A) = $ \dbox
\vfill


\clearpage
\newpage
\probpts{2}{10}  Suppose $A$ is an $m$ by $n$ matrix.  Show that the null space $N(A)$ and the row space $C(A^\top)$ are orthogonal, as subspaces of the vector space $\RR^n$.  (\emph{Hint.} What is a good way to write a generic vector from $C(A^\top)$?)
\vfill

\probpts{3}{6}  Suppose $A$ is an $m$ by $n$ matrix.  Show that $A^\top A$ is symmetric.
\vspace{3.0in}


\clearpage
\newpage
\prob{4}  Consider the overdetermined linear system ``$A\bv = \bb$'' with
    $$A = \begin{bmatrix} 1 & 1 \\ 0 & 2 \\ 1 & 2 \\ -1 & 1 \end{bmatrix}, \qquad \bb = \begin{bmatrix} 0 \\ 10 \\ -1 \\ -6 \end{bmatrix}.$$

\epartpts{a}{10}  Write down the normal equations for this system.
\vfill

\epartpts{b}{6}  The solution of the normal equations is $\bv = \begin{bmatrix} 1 \\ 1 \end{bmatrix}$.  Is $\bb$ in the column space $C(A)$ for this system?  How do you know?
\vspace{1.5in}


\clearpage
\newpage
\probpts{5}{10}  Consider this linear system $A\bx=\bb$:
\begin{align*}
x_1 + 2 x_2 + 2 x_3 + x_4 &= 9 \\
3 x_1 + 6 x_2 + 4 x_3 + x_4 &= 17
\end{align*}
Here is the row-reduced echelon form of the augmented matrix:
    $$[A\,\,\,\bb] \qquad \to \qquad
    [R\,\,\,\bd] = \begin{bmatrix} 1 & 2 & 0 & -1 & -1 \\ 0 & 0 & 1 & 1 & 5 \end{bmatrix}$$
What is the general solution of the system?  Show your work.
\vfill

\probpts{6}{6}  Show that if $A$ is any matrix and $\bx$ is in $N(A)$ then $\bx$ is in $N(A^\top A)$.
\vspace{2.5in}


\clearpage
\newpage
\probpts{7}{8}  $M_3$ is the vector space of all $3$ by $3$ matrices.  Give a basis for the subspace $S$ of symmetric matrices.
\vfill

\probpts{8}{6}  Suppose $A$ is any $m$ by $n$ matrix with full rank.  Let $P=A(A^\top A)^{-1} A^\top$, the projection onto $C(A)$.  Show that $P^2=P$.
\vfill

\probpts{9}{8}  Suppose $I$ is the $3$ by $3$ identity matrix and $O$ is the $2$ by $3$ zero matrix.  Consider the matrix
    $$A = \begin{bmatrix} I & I \\ O & O \end{bmatrix},$$
which is $5$ by $6$.  What are the dimensions of the four subspaces?
\begin{align*}
\dim C(A^\top) &= \mybox & \dim C(A) &= \mybox \\
\dim N(A) &= \mybox & \dim N(A^\top) &= \mybox
\end{align*}
\bigskip


\clearpage
\newpage
\prob{10}  Consider the line through the origin in $\RR^2$ along the vector $\ds \ba = \begin{bmatrix} 1 \\ 1 \end{bmatrix}$.  Suppose $\bb = \begin{bmatrix} -1 \\ 2 \end{bmatrix}$.

\medskip
\epartpts{a}{5}  Sketch $\ba$, the line along $\ba$, and the vector $\bb$, all on the same axes.  (\emph{Try to make your sketch to scale!})  Add $P\bb = \hat x \ba$, the projection of $\bb$ onto the line.
\vspace{4.0in}

\epartpts{b}{5}  Compute $\hat x$ and $P\bb = \hat x \ba$.
\vfill

\epartpts{c}{5}  Compute the projection matrix $P$.
\vfill


\clearpage
\newpage
\probpts{Extra Credit}{3}  The complete solution to $A\bx = \begin{bmatrix} 1 \\ 3 \end{bmatrix}$ is $\bx = \begin{bmatrix} 1 \\ 0 \end{bmatrix} + c \begin{bmatrix} 0 \\ 1 \end{bmatrix}$.  Find $A$.  Show your work.
% #33 in section 3.3
\vfill

\noindent \hrulefill
\begin{center}
\small
\textsc{blank space}
\end{center}
\vfill

\end{document}
