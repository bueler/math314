\documentclass[11pt]{amsart}
%\pagestyle{empty} 
\setlength{\topmargin}{-0.5in} % usually -0.25in
\addtolength{\textheight}{1.2in} % usually 1.25in
\addtolength{\oddsidemargin}{-0.95in}
\addtolength{\evensidemargin}{-0.95in}
\addtolength{\textwidth}{1.9in} %\setlength{\parindent}{0pt}

\newcommand{\normalspacing}{\renewcommand{\baselinestretch}{1.1}\tiny\normalsize}
\normalspacing

% macros
\usepackage{amssymb,xspace,alltt,verbatim}
\usepackage[final]{graphicx}
\usepackage[pdftex,colorlinks=true]{hyperref}
\usepackage{fancyvrb}
\usepackage{tikz}

\newtheorem*{lem*}{Lemma}

\newcommand{\bb}{\mathbf{b}}
\newcommand{\bc}{\mathbf{c}}
\newcommand{\bs}{\mathbf{s}}
\newcommand{\bu}{\mathbf{u}}
\newcommand{\bv}{\mathbf{v}}
\newcommand{\bw}{\mathbf{w}}
\newcommand{\bx}{\mathbf{x}}
\newcommand{\by}{\mathbf{y}}

\newcommand{\bbf}{\mathbf{f}}

\newcommand{\CC}{{\mathbb{C}}}
\newcommand{\RR}{{\mathbb{R}}}
\newcommand{\eps}{\epsilon}
\newcommand{\ZZ}{{\mathbb{Z}}}
\newcommand{\ZZn}{{\mathbb{Z}}_n}
\newcommand{\NN}{{\mathbb{N}}}
\newcommand{\ip}[2]{\mathrm{\left<#1,#2\right>}}

\renewcommand{\Re}{\operatorname{Re}}
\renewcommand{\Im}{\operatorname{Im}}

\newcommand{\Log}{\operatorname{Log}}

\newcommand{\grad}{\nabla}

\newcommand{\ds}{\displaystyle}

\newcommand{\Matlab}{\textsc{Matlab}\xspace}
\newcommand{\Octave}{\textsc{Octave}\xspace}
\newcommand{\pylab}{\textsc{pylab}\xspace}

\newcommand{\prob}[1]{\bigskip\noindent\textbf{#1.} }
\newcommand{\pts}[1]{(\emph{#1 pts})}

\newcommand{\probpts}[2]{\prob{#1} \pts{#2} \quad}
\newcommand{\ppartpts}[2]{\textbf{(#1)} \pts{#2} \quad}
\newcommand{\epartpts}[2]{\medskip\noindent \textbf{(#1)} \pts{#2} \quad}


\begin{document}
\hfill \Large Name:\underline{\phantom{Ed Bueler really really long long long name}}
\medskip

\scriptsize \noindent Math 314 Linear Algebra (Bueler) \hfill Monday, 14 February 2022
\medskip

\LARGE\centerline{\textbf{Midterm Exam 1}}

\smallskip
\begin{quote}
\large
\textbf{No book, notes, electronics, calculator, or internet access.  100 points possible. 65 minutes maximum.}
\end{quote}

\normalsize
\medskip

\thispagestyle{empty}

\prob{1}  Consider the following linear system $A\bx = \bb$:
\begin{align*}
2 x_1 + x_2 \phantom{+ 0x_3} &= 6 \\
-2 x_1 + 3 x_2 + 2 x_3 &= 0 \\
2 x_1 + 9 x_2 + 9 x_3 &= 13
\end{align*}
\begin{comment}
x =      2
         2
        -1
L =      1         0         0
        -1         1         0
         1         2         1
U =      2         1         0
         0         4         2
         0         0         5
\end{comment}

\epartpts{a}{10}  Solve the linear system by \emph{elimination} and then \emph{back-substitution}.  Use the standard algorithm.  Show your work, and in particular show, as an intermediate stage, the triangular system which you get after elimination.
\vfill

\clearpage
\newpage
\epartpts{b}{4}  From part \textbf{(a)}, what elimination matrix $E_{32}$ does the row operation which generated a zero in the $(3,2)$ location?
\vfill

\epartpts{c}{4}  From part \textbf{(a)}, what three numbers were the pivots?
\vfill

\epartpts{d}{6}  The computation in part \textbf{(a)} can regarded as factoring $A=LU$.  What lower triangular matrix $L$ and upper triangular matrix $U$ were computed?
\vfill

\epartpts{e}{2}  Multiply $LU$ and confirm you get the original matrix $A$.
\vfill

\clearpage
\newpage
\probpts{2}{10}  Suppose $A$ is an invertible $n\times n$ matrix which has a known LU factorization into a lower triangular matrix $L$ and an upper triangular matrix $U$.  That is, suppose $A=LU$.  Explain the steps, and name the algorithms, which you would use to solve a linear system $A\bx=\bb$.  (\emph{Hint.  The elimination stage has already been done.  Don't propose to redo it!})
\vfill

\probpts{3}{8}  I have claimed that
    $$\begin{bmatrix} a & b \\ c & d \end{bmatrix}^{-1} = \frac{1}{ad - bc} \begin{bmatrix} d & -b \\ -c & a \end{bmatrix}$$
Show (confirm) that this formula is correct.  (\emph{Hint.  A matrix multiplication suffices.})
\vfill

\newcommand{\tf}{\medskip\noindent\begin{tabular}{c} \textsc{true} \\ \textsc{false} \end{tabular}}
\clearpage
\newpage
\prob{4}  True or false?  \textbf{Circle one.}  Give a short justification if true, and a counterexample if false.

\epartpts{a}{3}  A matrix with two equal columns is not invertible.

\tf
\vfill

\epartpts{b}{3}  Every triangular matrix with 1's down the main diagonal is invertible.

\tf
\vfill

\epartpts{c}{3}  If $A$ is symmetric, so $a_{ij} = a_{ji}$, then $A$ is invertible.

\tf
\vfill

\epartpts{d}{3}  If $AB$ and $BA$ are defined then $A$ and $B$ are square.

\tf
\vfill

\epartpts{e}{3}  If $A$ and $B$ are square matrices of the same size then $(A+B)^2 = A^2 + 2AB + B^2$.

\tf
\vfill

\clearpage
\newpage
\prob{5} \ppartpts{a}{10}  Invert this matrix by the Gauss-Jordan method:
    $$A = \begin{bmatrix} 1 & 0 & -1 \\ 2 & 1 & 3 \\ 0 & 0 & 1 \end{bmatrix}$$
Please show your work!
\vfill
\begin{comment}
inv(A) =  1         0         1
         -2         1        -5
          0         0         1
\end{comment}

\epartpts{b}{2}  What is the determinant of the matrix in part \textbf{(a)}?
\vspace{1.0in}

\clearpage
\newpage
\probpts{6}{6}  Find the angle between these vectors.  (\emph{Hint.  You can write the answer in terms of an inverse trigonometric function, but otherwise everything should be simplified.})

    $$\bv = \begin{bmatrix} 2 \\ -1 \\ -2 \end{bmatrix} \quad \text{ and } \quad \bw = \begin{bmatrix} 1 \\ 1 \\ 1 \end{bmatrix} \hspace{5.0in}$$
\vfill

\probpts{7}{6}  Sketch the column picture of this linear system.  Noting that the solution values $(x,y)$ play a particular role in this picture, find them and then show the solution in the sketch:
\begin{align*}
x - 2 y &= 0 \\
x + y &= 3
\end{align*}
(\emph{Hint.  Clearly sketch and label the 3 known vectors.  Now, how to combine 2 of them to get the third?})
\vfill

\clearpage
\newpage
\prob{8}  Consider the linear system
\begin{align*}
a x - 2 y &= 1 \\
x + 4 y &= 3
\end{align*}

\epartpts{a}{4} For which number $a$ does elimination break down permanently, so that there are no solutions?
\vfill

\epartpts{b}{4} For which number $a$ does elimination break down temporarily, so that a row swap allows a solution?
\vfill

\epartpts{c}{3} For the value of $a$ found in part \textbf{(b)}, solve the system.
\vspace{2.0in}

\probpts{Extra Credit}{3}  Find the quadratic polynomial $p(x) = a + b x + c x^2$ which passes through the points $(-1,1)$, $(1,5)$, $(3,17)$.
% a=2, b=2, c=1
\vspace{2.0in}

\clearpage
\newpage
\probpts{9}{6}  Suppose $\ds L = \begin{bmatrix} 1 & 0 & 0 \\ a & 1 & 0 \\ b & c & 1 \end{bmatrix}$.  What is $L^{-1}$?
\vfill

\noindent \hrulefill
\begin{center}
\small
\textsc{blank space}
\end{center}
\vfill

\end{document}
