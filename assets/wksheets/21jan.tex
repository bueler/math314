\documentclass[11pt]{amsart}
%prepared in AMSLaTeX, under LaTeX2e
\addtolength{\oddsidemargin}{-.9in} 
\addtolength{\evensidemargin}{-.9in}
\addtolength{\topmargin}{-.9in}
\addtolength{\textwidth}{1.5in}
\addtolength{\textheight}{1.5in}

\renewcommand{\baselinestretch}{1.05}

\usepackage{verbatim} % for "comment" environment

\usepackage{palatino}

\usepackage[final]{graphicx}

\usepackage{tikz}
\usetikzlibrary{positioning}

\usepackage{enumitem,xspace,fancyvrb}

\newtheorem*{thm}{Theorem}
\newtheorem*{defn}{Definition}
\newtheorem*{example}{Example}
\newtheorem*{problem}{Problem}
\newtheorem*{remark}{Remark}

\DefineVerbatimEnvironment{mVerb}{Verbatim}{numbersep=2mm,frame=lines,framerule=0.1mm,framesep=2mm,xleftmargin=4mm,fontsize=\footnotesize}

% macros
\usepackage{amssymb}
\newcommand{\bA}{\mathbf{A}}
\newcommand{\bB}{\mathbf{B}}
\newcommand{\bE}{\mathbf{E}}
\newcommand{\bF}{\mathbf{F}}
\newcommand{\bJ}{\mathbf{J}}

\newcommand{\bb}{\mathbf{b}}
\newcommand{\br}{\mathbf{r}}
\newcommand{\bv}{\mathbf{v}}
\newcommand{\bw}{\mathbf{w}}
\newcommand{\bx}{\mathbf{x}}


\newcommand{\eps}{\epsilon}
\newcommand{\grad}{\nabla}
\newcommand{\ip}[2]{\ensuremath{\left<#1,#2\right>}}
\newcommand{\lam}{\lambda}
\newcommand{\lap}{\triangle}

\newcommand{\Null}{\operatorname{null}}
\newcommand{\rank}{\operatorname{rank}}
\newcommand{\range}{\operatorname{range}}
\newcommand{\trace}{\operatorname{tr}}

\newcommand{\RR}{\mathbb{R}}
\newcommand{\ZZ}{\mathbb{Z}}

\newcommand{\prob}[1]{\bigskip\noindent\textbf{#1.}\quad }
\newcommand{\exer}[2]{\prob{Exercise #2 on page #1}}

\newcommand{\pts}[1]{(\emph{#1 pts}) }
\newcommand{\epart}[1]{\bigskip\noindent\textbf{(#1)}\quad }
\newcommand{\ppart}[1]{\,\textbf{(#1)}\quad }

\newcommand{\Matlab}{\textsc{Matlab}\xspace}


\begin{document}
\scriptsize \noindent Math 314 Linear Algebra (Bueler) \hfill 21 January 2022 \fbox{\emph{Not to be turned in!}}
\normalsize\medskip

\Large\centerline{\textbf{Worksheet: 2 views of linear equations}}
\medskip
\normalsize

\thispagestyle{empty}
\begin{center}
Do these sketches and calculations with a group, if possible.
\end{center}

\prob{A}  Here are three equations in two unknowns:
\begin{align*}
2 x + 2 y &= 6 \\
x - 3 y &= -1 \\
4 x + y &= 0
\end{align*}
We can also write this as $A \bx = \bb$ where $A$ is a $3\times 2$ matrix and $\bx = \begin{bmatrix} x \\ y \end{bmatrix}$, $\bb = \begin{bmatrix} 6 \\ -1 \\ 0 \end{bmatrix}$ are vectors.

\epart{i} Sketch the ``row picture'': sketch each equation as a line in the $(x,y)$ plane.  Do they intersect?

\vfill
\epart{ii} Sketch the ``column picture'': sketch each column of $A$, and also $\bb$, in three-dimensional space.  Will you be able to find a linear combination of the columns of $A$ which gives $\bb$?

\vfill

\clearpage\newpage
\epart{iii}  Continuing problem \textbf{A}, change one entry of the right side so that the linear system does have a solution, and find that solution.

\vspace{2.5in}
\prob{B}  \ppart{i}  Consider the new linear system $A\bx=\bb$ where
    $$A = \begin{bmatrix} 1 & 0 & -1 \\ 2 & 2 & 0 \\ 3 & 2 & 1 \end{bmatrix}, \quad \bx = \begin{bmatrix} 0 \\ 4 \\ 6 \end{bmatrix},$$
and $\bx = (x_1,x_2,x_3)$ is unknown (\emph{for now}).  Sketch the row picture, that is, sketch each of the three equations as a plane.  Note it is easier to sketch each plane on separate axes; show three sketches.
\vfill

\epart{ii}  What is the solution of the system in part \textbf{(i)}?
\vspace{1.0in}
\end{document}
