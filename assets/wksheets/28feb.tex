\documentclass[12pt]{amsart}
%prepared in AMSLaTeX, under LaTeX2e
\addtolength{\oddsidemargin}{-1.1in} 
\addtolength{\evensidemargin}{-1.1in}
\addtolength{\topmargin}{-.9in}
\addtolength{\textwidth}{1.75in}
\addtolength{\textheight}{1.75in}

\renewcommand{\baselinestretch}{1.05}

\usepackage{verbatim} % for "comment" environment

\usepackage{palatino}

\usepackage[final]{graphicx}

\usepackage{tikz}
\usetikzlibrary{positioning}

\usepackage{bm,enumitem,xspace,fancyvrb}

\newtheorem*{thm}{Theorem}
\newtheorem*{defn}{Definition}
\newtheorem*{example}{Example}
\newtheorem*{problem}{Problem}
\newtheorem*{remark}{Remark}

\DefineVerbatimEnvironment{mVerb}{Verbatim}{numbersep=2mm,frame=lines,framerule=0.1mm,framesep=2mm,xleftmargin=4mm,fontsize=\footnotesize}

% macros
\usepackage{amssymb}
\newcommand{\bA}{\mathbf{A}}
\newcommand{\bB}{\mathbf{B}}
\newcommand{\bE}{\mathbf{E}}
\newcommand{\bF}{\mathbf{F}}
\newcommand{\bJ}{\mathbf{J}}

\newcommand{\bb}{\mathbf{b}}
\newcommand{\bc}{\mathbf{c}}
\newcommand{\bd}{\mathbf{d}}
\newcommand{\bi}{\mathbf{i}}
\newcommand{\bj}{\mathbf{j}}
\newcommand{\bk}{\mathbf{k}}
\newcommand{\br}{\mathbf{r}}
\newcommand{\bv}{\mathbf{v}}
\newcommand{\bw}{\mathbf{w}}
\newcommand{\bx}{\mathbf{x}}

\newcommand{\bzero}{\bm{0}}

\newcommand{\eps}{\epsilon}
\newcommand{\grad}{\nabla}
\newcommand{\ip}[2]{\ensuremath{\left<#1,#2\right>}}
\newcommand{\lam}{\lambda}
\newcommand{\lap}{\triangle}

\newcommand{\Null}{\operatorname{null}}
\newcommand{\rank}{\operatorname{rank}}
\newcommand{\range}{\operatorname{range}}
\newcommand{\trace}{\operatorname{tr}}

\newcommand{\RR}{\mathbf{R}}

\newcommand{\ZZ}{\mathbb{Z}}

\newcommand{\prob}[1]{\bigskip\noindent\textbf{#1.}\quad }
\newcommand{\exer}[2]{\prob{Exercise #2 on page #1}}

\newcommand{\pts}[1]{(\emph{#1 pts}) }
\newcommand{\epart}[1]{\bigskip\noindent\textbf{(#1)}\quad }
\newcommand{\ppart}[1]{\,\textbf{(#1)}\quad }

\newcommand{\Matlab}{\textsc{Matlab}\xspace}


\begin{document}
\scriptsize \noindent Math 314 Linear Algebra (Bueler) \hfill 28 February 2022 \fbox{\emph{Not to be turned in!}}
\normalsize\medskip

\Large\centerline{\textbf{Worksheet: Using the row-reduced echelon form}}
\medskip
\normalsize

\thispagestyle{empty}
\begin{quote}
For each linear system $A\bx=\bb$ below I applied Matlab's \texttt{rref()} command to the augmented matrix $[A \,\,\bb]$ to get the row-reduced echelon form $[R \,\,\bd]$.  Interpret it to answer the following questions:

\medskip
\begin{itemize}
\item what is the \textbf{rank} of $A$?
\item \textbf{find special solutions} which span the nullspace $N(A)$
\item \textbf{identify vectors} which span the column space $C(A)$
\item \textbf{write down} the general solution to the system $A\bx=\bb$
\end{itemize}
\end{quote}

\prob{1}
\begin{minipage}[t]{3in}
\begin{align*}
8 x_1 + x_2 + 15 x_3 &= -22 \\
3 x_1 + 5 x_2 + x_3 &= 1 \\
4 x_1 + 9 x_2 - x_3 &= 6
\end{align*}
\end{minipage}
\begin{minipage}[t]{1in}
\begin{align*}
&\phantom{x} \\
&\implies
\end{align*}
\end{minipage}
\begin{minipage}[t]{3in}
\medskip

\begin{align*}
&[R\,\,\bd] = \begin{bmatrix} 1 & 0 & 2 & -3 \\ 0 & 1 & -1 & 2 \\ 0 & 0 & 0 & 0 \end{bmatrix}
\end{align*}\end{minipage}
\vfill

\prob{2}  
\begin{minipage}[t]{3in}
\begin{align*}
12 x_1 - 10 x_2 + 5 x_3 &= -6 \\
- 9 x_1 - x_2 - 5 x_3 &= -32 \\
x_1 + 3 x_2 + 12 x_3 &= 38
\end{align*}
\end{minipage}
\begin{minipage}[t]{1in}
\begin{align*}
&\phantom{x} \\
&\implies
\end{align*}
\end{minipage}
\begin{minipage}[t]{3in}
\medskip

\begin{align*}
&[R\,\,\bd] = \begin{bmatrix} 1 & 0 & 0 & 2 \\ 0 & 1 & 0 & 4 \\ 0 & 0 & 1 & 2 \end{bmatrix}
\end{align*}\end{minipage}
\vfill

\clearpage \newpage
\prob{3}
\begin{minipage}[t]{3in}
\begin{align*}
2 x_1 - x_2 + 5 x_3 + 2 x_4 &= 5 \\
2 x_1 + x_2 - x_3 + 6 x_4 &= 7
\end{align*}
\end{minipage}
\begin{minipage}[t]{1in}
\begin{align*}
&\implies
\end{align*}
\end{minipage}
\begin{minipage}[t]{3in}
\medskip

\begin{align*}
&[R\,\,\bd] = \begin{bmatrix} 1 & 0 & 1 & 2 & 3 \\ 0 & 1 & -3 & 2 & 1 \end{bmatrix}
\end{align*}\end{minipage}
\vfill

\prob{4}
\begin{minipage}[t]{3in}
\begin{align*}
2 x_1 + 2 x_2 &= 4 \\
-x_1  + x_2   &= 4 \\
5 x_1 - x_2   &= -8 \\
2 x_1 + 6 x_2 &= 16
\end{align*}
\end{minipage}
\begin{minipage}[t]{1in}
\begin{align*}
&\implies
\end{align*}
\end{minipage}
\begin{minipage}[t]{3in}
\medskip

\begin{align*}
&[R\,\,\bd] = \begin{bmatrix} 1 & 0 & -1 \\ 0 & 1 & 3 \\ 0 & 0 & 0 \\ 0 & 0 & 0 \end{bmatrix}
\end{align*}\end{minipage} 
\vfill

\prob{5}  In problem \textbf{4} there are four equations in two unknowns.  In typical cases there would be no solutions at all.  Show a representative $[R \,\,\bd]$ when there are no solutions.
\vspace{1.5in}

\end{document}
