\documentclass[12pt]{amsart}
%prepared in AMSLaTeX, under LaTeX2e
\addtolength{\oddsidemargin}{-1.1in} 
\addtolength{\evensidemargin}{-1.1in}
\addtolength{\topmargin}{-.9in}
\addtolength{\textwidth}{1.75in}
\addtolength{\textheight}{1.75in}

\renewcommand{\baselinestretch}{1.05}

\usepackage{verbatim} % for "comment" environment

\usepackage{palatino}

\usepackage[final]{graphicx}

\usepackage{tikz}
\usetikzlibrary{positioning}

\usepackage{bm,enumitem,xspace,fancyvrb}

\newtheorem*{thm}{Theorem}
\newtheorem*{defn}{Definition}
\newtheorem*{example}{Example}
\newtheorem*{problem}{Problem}
\newtheorem*{remark}{Remark}

\DefineVerbatimEnvironment{mVerb}{Verbatim}{numbersep=2mm,frame=lines,framerule=0.1mm,framesep=2mm,xleftmargin=4mm,fontsize=\footnotesize}

% macros
\usepackage{amssymb}
\newcommand{\bA}{\mathbf{A}}
\newcommand{\bB}{\mathbf{B}}
\newcommand{\bE}{\mathbf{E}}
\newcommand{\bF}{\mathbf{F}}
\newcommand{\bJ}{\mathbf{J}}

\newcommand{\bb}{\mathbf{b}}
\newcommand{\bc}{\mathbf{c}}
\newcommand{\bd}{\mathbf{d}}
\newcommand{\bi}{\mathbf{i}}
\newcommand{\bj}{\mathbf{j}}
\newcommand{\bk}{\mathbf{k}}
\newcommand{\br}{\mathbf{r}}
\newcommand{\bv}{\mathbf{v}}
\newcommand{\bw}{\mathbf{w}}
\newcommand{\bx}{\mathbf{x}}

\newcommand{\bzero}{\bm{0}}

\newcommand{\eps}{\epsilon}
\newcommand{\grad}{\nabla}
\newcommand{\ip}[2]{\ensuremath{\left<#1,#2\right>}}
\newcommand{\lam}{\lambda}
\newcommand{\lap}{\triangle}

\newcommand{\Null}{\operatorname{null}}
\newcommand{\rank}{\operatorname{rank}}
\newcommand{\range}{\operatorname{range}}
\newcommand{\trace}{\operatorname{tr}}

\newcommand{\RR}{\mathbf{R}}

\newcommand{\ZZ}{\mathbb{Z}}

\newcommand{\prob}[1]{\bigskip\noindent\textbf{#1.}\quad }
\newcommand{\exer}[2]{\prob{Exercise #2 on page #1}}

\newcommand{\pts}[1]{(\emph{#1 pts}) }
\newcommand{\epart}[1]{\bigskip\noindent\textbf{(#1)}\quad }
\newcommand{\ppart}[1]{\,\textbf{(#1)}\quad }

\newcommand{\Matlab}{\textsc{Matlab}\xspace}
\newcommand{\ds}{\displaystyle}


\begin{document}
\scriptsize \noindent Math 314 Linear Algebra (Bueler) \hfill 11 April 2022 \fbox{\emph{Not to be turned in!}}
\normalsize\medskip

\Large\centerline{\textbf{Worksheet: eigenstuff, and diagonalizing matrices}}
\medskip
\normalsize

\thispagestyle{empty}
\begin{quote}
\textbf{(a)} \, For each matrix, compute the eigenvalues and eigenvectors by hand.  Confirm your result using computer assistance.

\medskip
\noindent \textbf{(b)} \, Is the matrix diagonalizable?  If it is, form a (convenient) invertible matrix $X$ of eigenvectors, and a diagonal matrix $\Lambda$ of eigenvalues

\medskip
\noindent \textbf{(c)} \, If the matrix was diagonalizable, confirm that $A = X \Lambda X^{-1}$.  This step may be done with computer assistance.
\end{quote}

\medskip

\prob{1} $\ds A = \begin{bmatrix} 1 & 1 & 1 \\ 0 & 2 & 1 \\ 0 & 0 & 3 \end{bmatrix}$
\vfill

\clearpage \newpage
\prob{2} $\ds B = \begin{bmatrix} 1 & 0 \\ 1 & 1 \end{bmatrix}$
\vspace{2.5in}

\prob{3} $\ds C = \begin{bmatrix} 1 & 0 & 1 \\ 0 & 1 & 0 \\ 1 & 0 & 1 \end{bmatrix}$
\vfill

\end{document}
