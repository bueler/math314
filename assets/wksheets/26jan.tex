\documentclass[11pt]{amsart}
%prepared in AMSLaTeX, under LaTeX2e
\addtolength{\oddsidemargin}{-.9in} 
\addtolength{\evensidemargin}{-.9in}
\addtolength{\topmargin}{-.9in}
\addtolength{\textwidth}{1.5in}
\addtolength{\textheight}{1.5in}

\renewcommand{\baselinestretch}{1.05}

\usepackage{verbatim} % for "comment" environment

\usepackage{palatino}

\usepackage[final]{graphicx}

\usepackage{tikz}
\usetikzlibrary{positioning}

\usepackage{enumitem,xspace,fancyvrb}

\newtheorem*{thm}{Theorem}
\newtheorem*{defn}{Definition}
\newtheorem*{example}{Example}
\newtheorem*{problem}{Problem}
\newtheorem*{remark}{Remark}

\DefineVerbatimEnvironment{mVerb}{Verbatim}{numbersep=2mm,frame=lines,framerule=0.1mm,framesep=2mm,xleftmargin=4mm,fontsize=\footnotesize}

% macros
\usepackage{amssymb}
\newcommand{\bA}{\mathbf{A}}
\newcommand{\bB}{\mathbf{B}}
\newcommand{\bE}{\mathbf{E}}
\newcommand{\bF}{\mathbf{F}}
\newcommand{\bJ}{\mathbf{J}}

\newcommand{\bb}{\mathbf{b}}
\newcommand{\bc}{\mathbf{c}}
\newcommand{\br}{\mathbf{r}}
\newcommand{\bv}{\mathbf{v}}
\newcommand{\bw}{\mathbf{w}}
\newcommand{\bx}{\mathbf{x}}


\newcommand{\eps}{\epsilon}
\newcommand{\grad}{\nabla}
\newcommand{\ip}[2]{\ensuremath{\left<#1,#2\right>}}
\newcommand{\lam}{\lambda}
\newcommand{\lap}{\triangle}

\newcommand{\Null}{\operatorname{null}}
\newcommand{\rank}{\operatorname{rank}}
\newcommand{\range}{\operatorname{range}}
\newcommand{\trace}{\operatorname{tr}}

\newcommand{\RR}{\mathbb{R}}
\newcommand{\ZZ}{\mathbb{Z}}

\newcommand{\prob}[1]{\bigskip\noindent\textbf{#1.}\quad }
\newcommand{\exer}[2]{\prob{Exercise #2 on page #1}}

\newcommand{\pts}[1]{(\emph{#1 pts}) }
\newcommand{\epart}[1]{\bigskip\noindent\textbf{(#1)}\quad }
\newcommand{\ppart}[1]{\,\textbf{(#1)}\quad }

\newcommand{\Matlab}{\textsc{Matlab}\xspace}


\begin{document}
\scriptsize \noindent Math 314 Linear Algebra (Bueler) \hfill 26 January 2022 \fbox{\emph{Not to be turned in!}}
\normalsize\medskip

\Large\centerline{\textbf{Worksheet: elimination, via row operations and matrices}}
\medskip
\normalsize

\thispagestyle{empty}
\begin{center}
Do these calculations with a group, if possible.
\end{center}

\prob{A}  Consider this system of three equations in three unknowns:
\begin{align*}
         x_1 + x_2 + x_3 &= 4 \\
      -8 x_1 + 4 x_2 \qquad &= -8 \\
       5 x_1 + 7 x_2 - 4 x_3 &= -7
\end{align*}
Do elimination, as described in lecture and in section 2.2, followed by back-substitution, to solve this system.  Make sure to follow the standard ordering of operations, even when it is tempting to do some \emph{ad hoc} steps!  Show your work in a reasonable way; documenting each step here will help in problem \textbf{B} on the other side.  (\emph{Hint.  The entries of $\bx$ are small integers.})
\vfill

\clearpage \newpage
\newcommand{\blankA}{\left[\begin{matrix} {\Huge \strut} \\ {\Huge \strut} \end{matrix} \phantom{lkdjfs adfk dsas dsff} \right]}
\newcommand{\blankb}{\left[\begin{matrix} {\Huge \strut} \\ {\Huge \strut} \end{matrix} \phantom{lkdf ed} \right]}

\prob{B}  Write the system in problem \textbf{A} (previous page) using matrices and vectors, as $A \bx = \bb$:
   $$A = \blankA, \qquad \bb = \blankb$$
Next, write down the three elimination matrices that correspond to the row operations you did.  That is, fill-in these $3\times 3$ matrices:
   $$E_{21} = \blankA, \quad E_{31} = \blankA, \quad E_{32} = \blankA$$
Finally, multiply-out matrices, as follows.  You should see that all the elimination steps on the previous page are reproduced here.
\begin{align*}
E_{21} A &= \blankA                   & E_{21} \bb &= \blankb \\
E_{31} (E_{21} A) &= \blankA          & E_{31} (E_{21} \bb) &= \blankb \\
E_{32} (E_{31} (E_{21} A)) &= \blankA & E_{32} (E_{31} (E_{21} \bb)) &= \blankb
\end{align*}
\vspace{1.0in}

\prob{C} Thus elimination converts the linear system to an upper triangular system:
    $$A\bx = \bb \qquad \longrightarrow \qquad U \bx = \bc$$
What are $U$ and $\bc$?:
   $$U = \blankA, \qquad \bc = \blankb$$
Do back-substitution to (again) solve the system and find $\bx$.
\vfill

\end{document}
