\documentclass[12pt]{amsart}
%prepared in AMSLaTeX, under LaTeX2e
\addtolength{\oddsidemargin}{-1.1in} 
\addtolength{\evensidemargin}{-1.1in}
\addtolength{\topmargin}{-.9in}
\addtolength{\textwidth}{1.75in}
\addtolength{\textheight}{1.75in}

\renewcommand{\baselinestretch}{1.05}

\usepackage{verbatim} % for "comment" environment

\usepackage{palatino}

\usepackage[final]{graphicx}

\usepackage{tikz}
\usetikzlibrary{positioning}

\usepackage{bm,enumitem,xspace,fancyvrb}

\newtheorem*{thm}{Theorem}
\newtheorem*{defn}{Definition}
\newtheorem*{example}{Example}
\newtheorem*{problem}{Problem}
\newtheorem*{remark}{Remark}

\DefineVerbatimEnvironment{mVerb}{Verbatim}{numbersep=2mm,frame=lines,framerule=0.1mm,framesep=2mm,xleftmargin=4mm,fontsize=\footnotesize}

% macros
\usepackage{amssymb}
\newcommand{\bA}{\mathbf{A}}
\newcommand{\bB}{\mathbf{B}}
\newcommand{\bE}{\mathbf{E}}
\newcommand{\bF}{\mathbf{F}}
\newcommand{\bJ}{\mathbf{J}}

\newcommand{\bb}{\mathbf{b}}
\newcommand{\bc}{\mathbf{c}}
\newcommand{\bd}{\mathbf{d}}
\newcommand{\bi}{\mathbf{i}}
\newcommand{\bj}{\mathbf{j}}
\newcommand{\bk}{\mathbf{k}}
\newcommand{\br}{\mathbf{r}}
\newcommand{\bv}{\mathbf{v}}
\newcommand{\bw}{\mathbf{w}}
\newcommand{\bx}{\mathbf{x}}

\newcommand{\bzero}{\bm{0}}

\newcommand{\eps}{\epsilon}
\newcommand{\grad}{\nabla}
\newcommand{\ip}[2]{\ensuremath{\left<#1,#2\right>}}
\newcommand{\lam}{\lambda}
\newcommand{\lap}{\triangle}

\newcommand{\Null}{\operatorname{null}}
\newcommand{\rank}{\operatorname{rank}}
\newcommand{\range}{\operatorname{range}}
\newcommand{\trace}{\operatorname{tr}}

\newcommand{\RR}{\mathbf{R}}

\newcommand{\ZZ}{\mathbb{Z}}

\newcommand{\prob}[1]{\bigskip\noindent\textbf{#1.}\quad }
\newcommand{\exer}[2]{\prob{Exercise #2 on page #1}}

\newcommand{\pts}[1]{(\emph{#1 pts}) }
\newcommand{\epart}[1]{\bigskip\noindent\textbf{(#1)}\quad }
\newcommand{\ppart}[1]{\,\textbf{(#1)}\quad }

\newcommand{\Matlab}{\textsc{Matlab}\xspace}
\newcommand{\ds}{\displaystyle}


\begin{document}
\scriptsize \noindent Math 314 Linear Algebra (Bueler) \hfill 16 March 2022 \fbox{\emph{Not to be turned in!}}
\normalsize\medskip

\Large\centerline{\textbf{Worksheet: The four subspaces}}
\medskip
\normalsize

\thispagestyle{empty}
\begin{quote}
For each matrix $A$ below I show $R=$\texttt{rref}$(A)$, i.e.~from Matlab.  Answer the following questions:

\medskip
\begin{itemize}
\item what are the dimensions of the four subspaces $C(A^\top)$, $C(A)$, $N(A)$, $N(A^\top)$?
\item find a basis for each of the first three subspaces $C(A^\top)$, $C(A)$, $N(A)$
\end{itemize}
\end{quote}

\prob{1}
    $$A = \begin{bmatrix} 8 & 1 & 15 \\ 3 & 5 & 1 \\ 4 & 9 & -1 \end{bmatrix} \qquad \to \qquad R = \begin{bmatrix} 1 & 0 & 2 \\ 0 & 1 & -1 \\ 0 & 0 & 0 \end{bmatrix}$$
\vfill

\prob{2}
    $$A = \begin{bmatrix} 12 & -10 & 5 \\ -9 & -1 & -5 \\ 1 & 3 & 12 \end{bmatrix} \qquad \to \qquad R = \begin{bmatrix} 1 & 0 & 0 \\ 0 & 1 & 0 \\ 0 & 0 & 1 \end{bmatrix}$$
\vfill

\clearpage \newpage
\prob{3}
    $$A = \begin{bmatrix} 2 & -1 & 5 & 2 \\ 2 & 1 & -1 & 6 \end{bmatrix} \qquad \to \qquad R = \begin{bmatrix} 1 & 0 & 1 & 2 \\ 0 & 1 & -3 & 2 \end{bmatrix}$$
\vfill

\prob{4}
    $$A = \begin{bmatrix} 2 & 2 \\ -1 & 1 \\ 5 & -1 \\ 2 & 6 \end{bmatrix} \qquad \to \qquad R = \begin{bmatrix} 1 & 0 \\ 0 & 1 \\ 0 & 0 \\ 0 & 0 \end{bmatrix}$$
\vfill

\prob{5} For $A$ in problem \textbf{4}, \texttt{rref}$\ds (A^\top)= \begin{bmatrix}
1 & 0 & 1 & 2 \\ 0 & 1 & -3 & 2 \end{bmatrix}$.  From this, find a basis for $N(A^\top)$.
\vfill
\end{document}
