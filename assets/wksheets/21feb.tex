\documentclass[12pt]{amsart}
%prepared in AMSLaTeX, under LaTeX2e
\addtolength{\oddsidemargin}{-1.1in} 
\addtolength{\evensidemargin}{-1.1in}
\addtolength{\topmargin}{-.9in}
\addtolength{\textwidth}{1.75in}
\addtolength{\textheight}{1.75in}

\renewcommand{\baselinestretch}{1.05}

\usepackage{verbatim} % for "comment" environment

\usepackage{palatino}

\usepackage[final]{graphicx}

\usepackage{tikz}
\usetikzlibrary{positioning}

\usepackage{bm,enumitem,xspace,fancyvrb}

\newtheorem*{thm}{Theorem}
\newtheorem*{defn}{Definition}
\newtheorem*{example}{Example}
\newtheorem*{problem}{Problem}
\newtheorem*{remark}{Remark}

\DefineVerbatimEnvironment{mVerb}{Verbatim}{numbersep=2mm,frame=lines,framerule=0.1mm,framesep=2mm,xleftmargin=4mm,fontsize=\footnotesize}

% macros
\usepackage{amssymb}
\newcommand{\bA}{\mathbf{A}}
\newcommand{\bB}{\mathbf{B}}
\newcommand{\bE}{\mathbf{E}}
\newcommand{\bF}{\mathbf{F}}
\newcommand{\bJ}{\mathbf{J}}

\newcommand{\bb}{\mathbf{b}}
\newcommand{\bc}{\mathbf{c}}
\newcommand{\bi}{\mathbf{i}}
\newcommand{\bj}{\mathbf{j}}
\newcommand{\bk}{\mathbf{k}}
\newcommand{\br}{\mathbf{r}}
\newcommand{\bv}{\mathbf{v}}
\newcommand{\bw}{\mathbf{w}}
\newcommand{\bx}{\mathbf{x}}

\newcommand{\bzero}{\bm{0}}

\newcommand{\eps}{\epsilon}
\newcommand{\grad}{\nabla}
\newcommand{\ip}[2]{\ensuremath{\left<#1,#2\right>}}
\newcommand{\lam}{\lambda}
\newcommand{\lap}{\triangle}

\newcommand{\Null}{\operatorname{null}}
\newcommand{\rank}{\operatorname{rank}}
\newcommand{\range}{\operatorname{range}}
\newcommand{\trace}{\operatorname{tr}}

\newcommand{\RR}{\mathbf{R}}

\newcommand{\ZZ}{\mathbb{Z}}

\newcommand{\prob}[1]{\bigskip\noindent\textbf{#1.}\quad }
\newcommand{\exer}[2]{\prob{Exercise #2 on page #1}}

\newcommand{\pts}[1]{(\emph{#1 pts}) }
\newcommand{\epart}[1]{\bigskip\noindent\textbf{(#1)}\quad }
\newcommand{\ppart}[1]{\,\textbf{(#1)}\quad }

\newcommand{\Matlab}{\textsc{Matlab}\xspace}


\begin{document}
\scriptsize \noindent Math 314 Linear Algebra (Bueler) \hfill 21 February 2022 \fbox{\emph{Not to be turned in!}}
\normalsize\medskip

\Large\centerline{\textbf{Worksheet: Is it a subspace?}}
\medskip
\normalsize

\thispagestyle{empty}
\begin{quote}
A \emph{vector space} is a set of vectors with a defined addition operation and a scalar multiple operation.  (Various sensible rules---p.~130---apply to those operations.)  A \emph{subspace} is a subset $S$ of the vector space for which any linear combination of elements from $S$ is in $S$.

You can verify that $S$ is a subspace by checking if $0$ is in $S$, if $\bv+\bw$ is in $S$ whenever $\bv$, $\bw$ are in $S$, and finally if $c\bv$ is in $S$ whenever $\bv$ is in $S$ and $c$ is any real number.

For each problem below, sketch $S$ if possible, and otherwise describe it.  Say whether $S$ is a subspace or not.  If so, provide a brief justification.  If not, describe an element that is not in $S$ but would be if $S$ were a subspace.
\end{quote}

\prob{1}  Vector space: $\RR^2$.  $S$ is the set of all points in the first quadrant of $\RR^2$.
\vfill

\prob{2}  Vector space: all real-valued functions on the line.  $S$ is the set of all polynomials.
\vfill

\prob{3}  Vector space: $\RR^3$.  $S$ is the set of all vectors $a\bi + b \bj + c\bk$ where $a,b,c$ are integers.
\vfill

\clearpage \newpage
\prob{4}  Vector space: $\RR^2$.  $S=$ all solutions to $A \bx = \bzero$ where $A=\begin{bmatrix} 2 & -3 \\ 6 & 7 \end{bmatrix}$.
\vfill

\prob{5}  Vector space: $\RR^2$.  $S=$ all solutions to $A \bx = \bb$ where $A=\begin{bmatrix} 2 & -3 \\ -6 & 9 \end{bmatrix}$ and $\bb = \begin{bmatrix}
-1 \\ 3 \end{bmatrix}$.
\vfill

\prob{6}  Vector space: $\RR^2$.  $S=$ all solutions to $A \bx = \bzero$ where $A=\begin{bmatrix} 2 & -3 \\ -6 & 9 \end{bmatrix}$.
\vfill

\prob{7}  Vector space: $\RR^3$.  $S=$ all solutions to $A \bx = \bzero$ where $A=\begin{bmatrix} 3 & 2 & 1 \\ 0 & 1 & 1 \\ -6 & -3 & -1 \end{bmatrix}$.
\vfill

\end{document}
