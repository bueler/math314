\documentclass[12pt]{amsart}
%prepared in AMSLaTeX, under LaTeX2e
\addtolength{\oddsidemargin}{-.9in} 
\addtolength{\evensidemargin}{-.9in}
\addtolength{\topmargin}{-.9in}
\addtolength{\textwidth}{1.5in}
\addtolength{\textheight}{1.5in}

\renewcommand{\baselinestretch}{1.05}

\usepackage{verbatim} % for "comment" environment

\usepackage{palatino}

\usepackage[final]{graphicx}

\usepackage{tikz}
\usetikzlibrary{positioning}

\usepackage{enumitem,xspace,fancyvrb}

\newtheorem*{thm}{Theorem}
\newtheorem*{defn}{Definition}
\newtheorem*{example}{Example}
\newtheorem*{problem}{Problem}
\newtheorem*{remark}{Remark}

\DefineVerbatimEnvironment{mVerb}{Verbatim}{numbersep=2mm,frame=lines,framerule=0.1mm,framesep=2mm,xleftmargin=4mm,fontsize=\footnotesize}

% macros
\usepackage{amssymb}
\newcommand{\bA}{\mathbf{A}}
\newcommand{\bB}{\mathbf{B}}
\newcommand{\bE}{\mathbf{E}}
\newcommand{\bF}{\mathbf{F}}
\newcommand{\bJ}{\mathbf{J}}

\newcommand{\bb}{\mathbf{b}}
\newcommand{\bc}{\mathbf{c}}
\newcommand{\br}{\mathbf{r}}
\newcommand{\bv}{\mathbf{v}}
\newcommand{\bw}{\mathbf{w}}
\newcommand{\bx}{\mathbf{x}}


\newcommand{\eps}{\epsilon}
\newcommand{\grad}{\nabla}
\newcommand{\ip}[2]{\ensuremath{\left<#1,#2\right>}}
\newcommand{\lam}{\lambda}
\newcommand{\lap}{\triangle}

\newcommand{\Null}{\operatorname{null}}
\newcommand{\rank}{\operatorname{rank}}
\newcommand{\range}{\operatorname{range}}
\newcommand{\trace}{\operatorname{tr}}

\newcommand{\RR}{\mathbb{R}}
\newcommand{\ZZ}{\mathbb{Z}}

\newcommand{\prob}[1]{\bigskip\noindent\textbf{#1.}\, }

\newcommand{\epart}[1]{\bigskip\noindent\textbf{(#1)}\, }
\newcommand{\ppart}[1]{\,\textbf{(#1)}\, }

\newcommand{\Matlab}{\textsc{Matlab}\xspace}


\begin{document}
\scriptsize \noindent Math 314 Linear Algebra (Bueler) \hfill 10 February 2022 \fbox{\emph{Not to be turned in!}}
\normalsize\medskip

\Large\centerline{\textbf{Worksheet: From linear system to LU factorization}}
\medskip
\normalsize

\thispagestyle{empty}
\begin{center}
Do these problems with a group, if possible!
\end{center}

\prob{I}  Consider the following linear system $A\bx = \bb$:
\begin{align*}
2 x_1 + x_2 - 9 x_3 &= -6 \\
4 x_1 - 3 x_2 -21 x_3 &= -20 \\
-6 x_1 - 13 x_2 + 24 x_3 &= 5
\end{align*}
\begin{comment}
L =      1         0         0
         2         1         0
        -3         2         1
U =      2         1        -9
         0        -5        -3
         0         0         3
>> A = L*U
A =      2         1        -9
         4        -3       -21
        -6       -13        24
>> x = [1 1 1]';
>> b = A*x
b =     -6
       -20
         5
\end{comment}
Solve the linear system by \emph{elimination} and then \emph{back-substitution}.  Use the standard algorithm.  Show your work, and in particular show, as an intermediate stage, the triangular system which you get after elimination.
\vfill

\clearpage
\newpage
\prob{II}  \ppart{a}  From problem \textbf{I}, what elimination matrices $E_{21},E_{31}E_{32}$ did the row operations?
\vfill

\epart{b}  What are the inverses $L_{21}=E_{21}^{-1}$, $L_{31}=E_{31}^{-1}$, $L_{32}=E_{32}^{-1}$?
\vfill

\epart{c}  From problem \textbf{I}, what numbers were the pivots?  What is the determinant of A?
\vfill

\epart{d}  The computation in problem \textbf{I} can regarded as factoring $A=LU$.  What lower triangular matrix $L$ and upper triangular matrix $U$ were computed?
\vfill

\epart{e}  Multiply $LU$ and confirm you get the original matrix $A$.
\vspace{0.5in}


%\noindent \hrule
%\bigskip
%\centerline{\footnotesize \textsc{blank space}}
%\vspace{4.0in}
\end{document}
