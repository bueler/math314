\documentclass[12pt]{amsart}
%prepared in AMSLaTeX, under LaTeX2e
\addtolength{\oddsidemargin}{-.9in} 
\addtolength{\evensidemargin}{-.9in}
\addtolength{\topmargin}{-.9in}
\addtolength{\textwidth}{1.5in}
\addtolength{\textheight}{1.5in}

\renewcommand{\baselinestretch}{1.05}

\usepackage{verbatim} % for "comment" environment

\usepackage{palatino}

\usepackage[final]{graphicx}

\usepackage{tikz}
\usetikzlibrary{positioning}

\usepackage{enumitem,xspace,fancyvrb}

\newtheorem*{thm}{Theorem}
\newtheorem*{defn}{Definition}
\newtheorem*{example}{Example}
\newtheorem*{problem}{Problem}
\newtheorem*{remark}{Remark}

\DefineVerbatimEnvironment{mVerb}{Verbatim}{numbersep=2mm,frame=lines,framerule=0.1mm,framesep=2mm,xleftmargin=4mm,fontsize=\footnotesize}

% macros
\usepackage{amssymb}
\newcommand{\bA}{\mathbf{A}}
\newcommand{\bB}{\mathbf{B}}
\newcommand{\bE}{\mathbf{E}}
\newcommand{\bF}{\mathbf{F}}
\newcommand{\bJ}{\mathbf{J}}

\newcommand{\bb}{\mathbf{b}}
\newcommand{\bc}{\mathbf{c}}
\newcommand{\br}{\mathbf{r}}
\newcommand{\bv}{\mathbf{v}}
\newcommand{\bw}{\mathbf{w}}
\newcommand{\bx}{\mathbf{x}}


\newcommand{\eps}{\epsilon}
\newcommand{\grad}{\nabla}
\newcommand{\ip}[2]{\ensuremath{\left<#1,#2\right>}}
\newcommand{\lam}{\lambda}
\newcommand{\lap}{\triangle}

\newcommand{\Null}{\operatorname{null}}
\newcommand{\rank}{\operatorname{rank}}
\newcommand{\range}{\operatorname{range}}
\newcommand{\trace}{\operatorname{tr}}

\newcommand{\RR}{\mathbb{R}}
\newcommand{\ZZ}{\mathbb{Z}}

\newcommand{\prob}[1]{\bigskip\noindent\textbf{#1.}\quad }
\newcommand{\exer}[2]{\prob{Exercise #2 on page #1}}

\newcommand{\pts}[1]{(\emph{#1 pts}) }
\newcommand{\epart}[1]{\bigskip\noindent\textbf{(#1)}\quad }
\newcommand{\ppart}[1]{\,\textbf{(#1)}\quad }

\newcommand{\Matlab}{\textsc{Matlab}\xspace}


\begin{document}
\scriptsize \noindent Math 314 Linear Algebra (Bueler) \hfill 2 February 2022 \fbox{\emph{Not to be turned in!}}
\normalsize\medskip

\Large\centerline{\textbf{Worksheet: getting comfortable with matrix indexing}}
\medskip
\normalsize

\thispagestyle{empty}
\begin{center}
All of these problems use ``$a_{ij}$'' for the entry in row $i$ and column $j$ of a matrix $A$. 

Do these problems with a group, if possible!
\end{center}

\prob{I}  Write down \textbf{the} 3 by 3 matrix $A$ whose entries are given by

\epart{a} $a_{ij} = \text{minimum of } i \text{ and } j$
\vfill

\epart{b} $a_{ij} = (-1)^{i+j}$
\vfill

\epart{c} $a_{ij} = i/j$
\vfill

\prob{II}  What words would you use to describe each of these classes of matrices?  Give \textbf{a} 3 by 3 example in each class.  Which matrix belongs to all four classes?

\epart{a} $a_{ij} = 0 \text{ if } i \ne j$
\vfill

\epart{b} $a_{ij} = 0 \text{ if } i < j$
\vfill

\epart{c} $a_{ij} = a_{ji}$
\vfill

\epart{d} $a_{ij} = a_{1j}$
\vfill

\clearpage \newpage
\prob{III}  Write a sum in terms of entries $a_{ij}$ and $b_{ij}$.  \textbf{Don't} look-up the formula!

\epart{a} Suppose $A$ is $m \times n$ and $B$ is $n \times p$.  Let $C=AB$.  Write a formula for each entry of $C$:

$$c_{ij} = \hspace{6.0in}$$
\vfill

\epart{b} Suppose $A$ is $m \times 1$ and $B$ is $1 \times p$, so $A$ is actually a column vector and $B$ is actually a row vector.  Again let $C=AB$, and write a formula for $c_{ij}$.  (This time it is very simple!  You are writing-out how the \emph{outer product} works.)

$$c_{ij} = \hspace{6.0in}$$
\vspace{1.0in}

\noindent \hrule
\bigskip
\centerline{\footnotesize \textsc{blank space}}
\vspace{4.0in}
\end{document}
