\documentclass[12pt]{amsart}
%prepared in AMSLaTeX, under LaTeX2e
\addtolength{\oddsidemargin}{-.65in} 
\addtolength{\evensidemargin}{-.65in}
\addtolength{\topmargin}{-.4in}
\addtolength{\textwidth}{1.4in}
\addtolength{\textheight}{0.8in}

\renewcommand{\baselinestretch}{1.05}

\usepackage{xspace,bm}
\usepackage{verbatim,fancyvrb}
\usepackage{palatino}
\usepackage[final]{graphicx}
\usepackage[pdftex,colorlinks=True]{hyperref}

\newtheorem*{thm}{Theorem}
\newtheorem*{defn}{Definition}
\newtheorem*{example}{Example}
\newtheorem*{problem}{Problem}
\newtheorem*{remark}{Remark}

\newcommand{\mfile}[1]{
\begin{quote}
\bigskip
\VerbatimInput[frame=single,framesep=3mm,label=\fbox{\normalsize \textsl{\,#1\,}},fontfamily=courier,fontsize=\footnotesize]{../matlab/#1}
\medskip
\end{quote}
}

%\DefineVerbatimEnvironment{mVerb}{Verbatim}{numbersep=2mm,frame=lines,framerule=0.1mm,framesep=2mm,xleftmargin=4mm,fontsize=\footnotesize}
\DefineVerbatimEnvironment{mVerb}{Verbatim}{numbersep=2mm,xleftmargin=4mm,fontsize=\footnotesize}

% macros
\usepackage{amssymb}

\newcommand{\ba}{\bm{a}}
\newcommand{\bb}{\bm{b}}
\newcommand{\bc}{\bm{c}}
\newcommand{\br}{\bm{r}}
\newcommand{\bu}{\bm{u}}
\newcommand{\bv}{\bm{v}}
\newcommand{\bw}{\bm{w}}
\newcommand{\bx}{\bm{x}}
\newcommand{\by}{\bm{y}}
\newcommand{\bz}{\bm{z}}

\newcommand{\CC}{\mathbb{C}}
\newcommand{\RR}{\mathbb{R}}
\newcommand{\ZZ}{\mathbb{Z}}

\newcommand{\eps}{\epsilon}
\newcommand{\grad}{\nabla}
\newcommand{\lam}{\lambda}
\newcommand{\lap}{\triangle}

\newcommand{\ip}[2]{\ensuremath{\left<#1,#2\right>}}

\newcommand{\image}{\operatorname{im}}
\newcommand{\onull}{\operatorname{null}}
\newcommand{\rank}{\operatorname{rank}}
\newcommand{\range}{\operatorname{range}}
\newcommand{\trace}{\operatorname{tr}}

\newcommand{\ds}{\displaystyle}

\newcommand{\prob}[1]{\bigskip\noindent\textbf{#1}\quad }

\newcommand{\probset}[2]{\prob{from Problem Set #1, pages #2:}}

\newcommand{\pts}[1]{(\emph{#1 pts}) }
\newcommand{\epart}[1]{\medskip\noindent\textbf{(#1)}\quad }
\newcommand{\ppart}[1]{\,\textbf{(#1)}\quad }

\newcommand{\Matlab}{\textsc{Matlab}\xspace}
\newcommand{\Octave}{\textsc{Octave}\xspace}


\begin{document}
\scriptsize \noindent Math 314 Linear Algebra (Bueler) \hfill Spring 2022
\normalsize\medskip

\Large
\centerline{\textbf{Homework \#3}}

\normalsize
\bigskip
\begin{quote}
Due Monday 31 January, 2022 at 11:59pm.

\medskip
\noindent Submit as a single PDF by using Gradescope, via

\href{https://canvas.alaska.edu/courses/7017}{\texttt{canvas.alaska.edu/courses/7017}}

\noindent Problems from the textbook (Strang, \emph{Intro Linear Algebra}, 5th ed.~2016) will be graded for completion, while the ``\textbf{P}'' Problems will be graded for correctness.  Answers/solutions to Textbook Problems are linked at

\href{https://bueler.github.io/math314/resources.html}{\texttt{bueler.github.io/math314/resources.html}}
\end{quote}
\medskip

\thispagestyle{empty}

\noindent \hrulefill

\noindent \emph{Put these Textbook Problems first on your PDF, in this order.}

\probset{2.2}{53--57} \# 1, 4, 6, 9, 11, 12, 14, 21, 27

\probset{2.3}{66--69} \# 1, 3, 4, 12, 19, 24, 28, 29


\bigskip
\noindent \hrulefill

\noindent \emph{Put these ``P'' Problems next on your PDF, in this order.}

\prob{P11.}  \ppart{a} Complete the right side to get a system which has no solutions:
\begin{align*}
8 x - 4 y &= 14 \\
-2 x + y &= \fbox{\strut \quad}
\end{align*}
(\emph{There are many correct answers.})

\epart{b} Complete the right side to get a system which has infinitely-many solutions.  (\emph{There is only one correct answer.})

\epart{c} Write down two different solutions to the system in part \textbf{(b)}.

\prob{P12.}  Find three possible original problems (linear systems) so that elimination leads to $x-y=1$ and $2y = -3$.

\prob{P13.}  Suppose we start with some 4 by 4 matrix $A$.

\epart{a} $E_{21}$ subtracts row 1 from row 2 and then $P_{24}$ exchanges rows 2 and 4.  What matrix $M=P_{24} E_{21}$ does both steps at once?

\epart{b} $P_{24}$ exchanges rows 2 and 4 and then $E_{41}$ subtracts row 1 from row 4. What matrix $N=E_{41} P_{24}$ does both steps at once?

\epart{c} Explain why $M=N$.

\prob{P14.}  Elimination on this 4 by 4 matrix $Z$ will need matrices $E_{21}$, $E_{32}$, and $E_{43}$.  What are these matrices?
    $$Z = \begin{bmatrix} 2 & -1 & 0 & 0 \\ 1 & 2 & -1 & 0 \\ 0 & 1 & 2 & -1 \\ 0 & 0 & 1 & 3 \end{bmatrix}$$

\prob{P15.}  Consider the matrices
    $$A = \begin{bmatrix} 1 & 0 & 0 \\ a & 1 & 0 \\ 0 & 0 & 1 \end{bmatrix}, \qquad B = \begin{bmatrix} 1 & 0 & 0 \\ 0 & 1 & 0 \\ 0 & b & 1 \end{bmatrix}.$$

\epart{a} Compute $AB$ and $BA$.  Are they the same?

\epart{b} Give formulas for $A^n$ and $B^n$ if $n\ge 1$ is an integer.  (\emph{Hint.} Compute $A^2,A^3$ and $B^2,B^3$.  What is the pattern?)

\prob{P16.}  Consider this system $Ax=\bb$:
    $$\begin{bmatrix} 1 & 2 & 3 \\ 2 & 5 & 1 \\ 3 & 7 & 4 \end{bmatrix} \begin{bmatrix} x_1 \\ x_2 \\ x_3 \end{bmatrix} = \begin{bmatrix} 1 \\ 2 \\ 6 \end{bmatrix}$$

\epart{a}  Apply elimination to the augmented matrix $\left[A\,\, \bb\right]$.  How do you know this system has no solution?

\epart{b}  Change the last number 6 so that the new system does have a solution.  Find a solution of the new system.

\end{document}
