\documentclass[12pt]{amsart}
%prepared in AMSLaTeX, under LaTeX2e
\addtolength{\oddsidemargin}{-.65in}
\addtolength{\evensidemargin}{-.65in}
\addtolength{\topmargin}{-.4in}
\addtolength{\textwidth}{1.4in}
\addtolength{\textheight}{0.8in}

\renewcommand{\baselinestretch}{1.05}

\usepackage{xspace,bm}
\usepackage{verbatim,fancyvrb}
\usepackage{palatino}
\usepackage[final]{graphicx}
\usepackage[pdftex,colorlinks=True]{hyperref}

\newtheorem*{thm}{Theorem}
\newtheorem*{defn}{Definition}
\newtheorem*{example}{Example}
\newtheorem*{problem}{Problem}
\newtheorem*{remark}{Remark}

\newcommand{\mfile}[1]{
\begin{quote}
\bigskip
\VerbatimInput[frame=single,framesep=3mm,label=\fbox{\normalsize \textsl{\,#1\,}},fontfamily=courier,fontsize=\footnotesize]{../matlab/#1}
\medskip
\end{quote}
}

%\DefineVerbatimEnvironment{mVerb}{Verbatim}{numbersep=2mm,frame=lines,framerule=0.1mm,framesep=2mm,xleftmargin=4mm,fontsize=\footnotesize}
\DefineVerbatimEnvironment{mVerb}{Verbatim}{numbersep=2mm,xleftmargin=4mm,fontsize=\footnotesize}

% macros
\usepackage{amssymb}

\newcommand{\ba}{\bm{a}}
\newcommand{\bb}{\bm{b}}
\newcommand{\bc}{\bm{c}}
\newcommand{\bp}{\bm{p}}
\newcommand{\br}{\bm{r}}
\newcommand{\bu}{\bm{u}}
\newcommand{\bv}{\bm{v}}
\newcommand{\bw}{\bm{w}}
\newcommand{\bx}{\bm{x}}
\newcommand{\by}{\bm{y}}
\newcommand{\bz}{\bm{z}}

\newcommand{\bR}{\bm{R}}
\newcommand{\bS}{\bm{S}}
\newcommand{\bV}{\bm{V}}

\newcommand{\bzero}{\bm{0}}

\newcommand{\CC}{\mathbb{C}}
\newcommand{\RR}{\mathbb{R}}
\newcommand{\ZZ}{\mathbb{Z}}

\newcommand{\eps}{\epsilon}
\newcommand{\grad}{\nabla}
\newcommand{\lam}{\lambda}
\newcommand{\lap}{\triangle}

\newcommand{\ip}[2]{\ensuremath{\left<#1,#2\right>}}

\newcommand{\image}{\operatorname{im}}
\newcommand{\onull}{\operatorname{null}}
\newcommand{\rank}{\operatorname{rank}}
\newcommand{\range}{\operatorname{range}}
\newcommand{\trace}{\operatorname{tr}}

\newcommand{\ds}{\displaystyle}

\newcommand{\prob}[1]{\bigskip\noindent\textbf{#1.}\quad }
\newcommand{\probset}[2]{\bigskip\noindent\textbf{from Problem Set #1, pages #2:}\quad }

\newcommand{\pts}[1]{(\emph{#1 pts}) }
\newcommand{\epart}[1]{\medskip\noindent\textbf{(#1)}\quad }
\newcommand{\ppart}[1]{\,\textbf{(#1)}\quad }

\newcommand{\Matlab}{\textsc{Matlab}\xspace}
\newcommand{\Octave}{\textsc{Octave}\xspace}


\begin{document}
\scriptsize \noindent Math 314 Linear Algebra (Bueler) \hfill Spring 2022
\normalsize\medskip

\Large
\centerline{\textbf{Homework \#9}}

\bigskip
\large
\centerline{Due Friday 25 March, 2022 at 11:59pm.}

\normalsize
\bigskip
\begin{quote}
\medskip
\noindent Submit as a single PDF via Gradescope; see the Canvas page

\href{https://canvas.alaska.edu/courses/7017}{\texttt{canvas.alaska.edu/courses/7017}}

\noindent Textbook Problems from Strang, \emph{Intro Linear Algebra}, 5th ed.~will be graded for completion.  Answers/solutions to these Problems are linked at

\href{https://bueler.github.io/math314/resources.html}{\texttt{bueler.github.io/math314/resources.html}}

\noindent The \textbf{P} Problems will be graded for correctness.  When grading these Problems, I will expect you to write explanations using complete sentences!
\end{quote}
\medskip

\thispagestyle{empty}

\noindent \hrulefill

\noindent \emph{Put these Textbook Problems first on your PDF, in this order.}

\probset{4.2}{213--217} \# 1, 3, 8, 13, 16, 17, 21, 22

\probset{4.3}{228--231} \# 1, 2, 3, 4, 8, 9


\bigskip
\noindent \hrulefill

\noindent \emph{Put these \textbf{P} Problems next on your PDF, in this order.}

\prob{P43}  For each part: \emph{i)} Draw the projection $\bp$ of $\bb$ onto $\ba$.  \emph{ii)} Compute it as $\bp = \hat x \ba$, where $\ds \hat x = \frac{\ba^\top \bb}{\ba^\top \ba}$.  \emph{iii)} Compute the projection matrix $\ds P = \frac{\ba \ba^\top}{\ba^\top \ba}$, and then $\bp = P\bb$.

\epart{a}  $\ds \bb = \begin{bmatrix} 1 \\ 0 \end{bmatrix}$ \, and \, $\ds \ba = \begin{bmatrix} 1 \\ 1 \end{bmatrix}$

\epart{b}  $\ds \bb = \begin{bmatrix} 1 \\ -1 \end{bmatrix}$ \, and \, $\ds \ba = \begin{bmatrix} 1 \\ 1 \end{bmatrix}$

\epart{c}  $\ds \bb = \begin{bmatrix} 1 \\ 1 \end{bmatrix}$ \, and \, $\ds \ba = \begin{bmatrix} \cos\theta \\ \sin\theta \end{bmatrix}$ \hfill  (\emph{For your drawing, just pick a generic $\theta$.})


\prob{P44}  For each part: \emph{i)} Form and solve the normal equations $A^\top A \hat\bx = A^\top \bb$.  \emph{ii)} Compute the projection matrix $\ds P = A (A^\top A)^{-1} A^\top$.  (\emph{You can use technology for the inverse.}) \emph{iii)} Check that $P^2=P$ and $P^\top=P$.  \emph{iv)} Compute $\bp = P \bb$, and check it matches $A\hat \bx$ from the solution to the normal equations.

\epart{a}  $\ds A = \begin{bmatrix} 1 & 1 \\ 0 & 1 \\ 0 & 0 \end{bmatrix}$ \, and \, $\ds \bb = \begin{bmatrix} 2 \\ 3 \\ 4 \end{bmatrix}$

\epart{b}  $\ds A = \begin{bmatrix} 1 & 1 \\ 1 & 2 \\ 0 & 1 \end{bmatrix}$ \, and \, $\ds \bb = \begin{bmatrix} 2 \\ 8 \\ 6 \end{bmatrix}$



\prob{P45}  An overdetermined system you cannot solve (4 equations in 2 unknowns):
\begin{align*}
x_1 + x_2 &= 1 \\
x_1 \qquad &= 0 \\
2 x_1 - x_2 &= 2 \\
3 x_1 + 4 x_2 &= -1
\end{align*}

\epart{a} Each equation is a line in the $x_1,x_2$ plane.  Plot all 4 lines in one plot.  They do not meet in a single point.  (\emph{Feel free to use technology for this plot.  Your plot box should at least include all the places where pairs of lines intersect.})
% Wolfram alpha:  plot x+y=1, x=0, 2x-y=2, 3x+4y=-1
% desmos plot easy

\epart{b} Write down $A$ and $\bb$ for the above system ``$A\bx = \bb$''.  Now form $A^\top A$ and $A^\top \bb$.

\epart{c} Solve the normal equations $A^\top A \bx = A^\top \bb$.  (\emph{Use technology as desired.})  Add the solution point to your plot in part \textbf{(a)}.


\prob{P46}  \ppart{a}  Consider the same $A,\bb$ as in P45.  Write out and simplify
    $$E(x_1,x_2) = \|A\bx - \bb\|^2 = (A\bx - \bb)^\top (A\bx - \bb).$$
(\emph{Hint.  This simplifies to a function which is quadratic in the two variables $x_1,x_2$.})

\epart{b}  Compute and simplify the partial derivatives of $E$.

\epart{c}  Solve the linear system of two equations in two unknowns $x_1,x_2$:
\begin{align*}
\frac{\partial E}{\partial x_1} &= 0 \\
\frac{\partial E}{\partial x_2} &= 0
\end{align*}
(\emph{Hint.  The solution is the same as in \textbf{P45 (c)}.  The system is essentially the same.})

\end{document}
