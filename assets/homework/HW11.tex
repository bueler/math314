\documentclass[12pt]{amsart}
%prepared in AMSLaTeX, under LaTeX2e
\addtolength{\oddsidemargin}{-.65in}
\addtolength{\evensidemargin}{-.65in}
\addtolength{\topmargin}{-.4in}
\addtolength{\textwidth}{1.4in}
\addtolength{\textheight}{0.8in}

\renewcommand{\baselinestretch}{1.05}

\usepackage{xspace,bm}
\usepackage{verbatim,fancyvrb}
\usepackage{palatino}
\usepackage[final]{graphicx}
\usepackage[pdftex,colorlinks=True]{hyperref}

\newtheorem*{thm}{Theorem}
\newtheorem*{defn}{Definition}
\newtheorem*{example}{Example}
\newtheorem*{problem}{Problem}
\newtheorem*{remark}{Remark}

\newcommand{\mfile}[1]{
\begin{quote}
\bigskip
\VerbatimInput[frame=single,framesep=3mm,label=\fbox{\normalsize \textsl{\,#1\,}},fontfamily=courier,fontsize=\footnotesize]{../matlab/#1}
\medskip
\end{quote}
}

%\DefineVerbatimEnvironment{mVerb}{Verbatim}{numbersep=2mm,frame=lines,framerule=0.1mm,framesep=2mm,xleftmargin=4mm,fontsize=\footnotesize}
\DefineVerbatimEnvironment{mVerb}{Verbatim}{numbersep=2mm,xleftmargin=4mm,fontsize=\footnotesize}

% macros
\usepackage{amssymb}

\newcommand{\ba}{\bm{a}}
\newcommand{\bb}{\bm{b}}
\newcommand{\bc}{\bm{c}}
\newcommand{\bp}{\bm{p}}
\newcommand{\br}{\bm{r}}
\newcommand{\bu}{\bm{u}}
\newcommand{\bv}{\bm{v}}
\newcommand{\bw}{\bm{w}}
\newcommand{\bx}{\bm{x}}
\newcommand{\by}{\bm{y}}
\newcommand{\bz}{\bm{z}}

\newcommand{\bR}{\bm{R}}
\newcommand{\bS}{\bm{S}}
\newcommand{\bV}{\bm{V}}

\newcommand{\bzero}{\bm{0}}

\newcommand{\CC}{\mathbb{C}}
\newcommand{\RR}{\mathbb{R}}
\newcommand{\ZZ}{\mathbb{Z}}

\newcommand{\eps}{\epsilon}
\newcommand{\grad}{\nabla}
\newcommand{\lam}{\lambda}
\newcommand{\lap}{\triangle}

\newcommand{\ip}[2]{\ensuremath{\left<#1,#2\right>}}

\newcommand{\image}{\operatorname{im}}
\newcommand{\onull}{\operatorname{null}}
\newcommand{\rank}{\operatorname{rank}}
\newcommand{\range}{\operatorname{range}}
\newcommand{\trace}{\operatorname{tr}}

\newcommand{\ds}{\displaystyle}

\newcommand{\prob}[1]{\bigskip\noindent\textbf{#1.}\quad }
\newcommand{\probset}[2]{\bigskip\noindent\textbf{from Problem Set #1, pages #2:}\quad }

\newcommand{\pts}[1]{(\emph{#1 pts}) }
\newcommand{\epart}[1]{\medskip\noindent\textbf{(#1)}\quad }
\newcommand{\ppart}[1]{\,\textbf{(#1)}\quad }

\newcommand{\Matlab}{\textsc{Matlab}\xspace}
\newcommand{\Octave}{\textsc{Octave}\xspace}


\begin{document}
\scriptsize \noindent Math 314 Linear Algebra (Bueler) \hfill 29 March 2022
\normalsize\medskip

\Large
\centerline{\textbf{Homework \#11}}

\bigskip
\large
\centerline{Due Wednesday 13 April, 2022 at 11:59pm.}

\normalsize
\bigskip
\begin{quote}
\medskip
\noindent Submit as a single PDF via Gradescope; see the Canvas page

\href{https://canvas.alaska.edu/courses/7017}{\texttt{canvas.alaska.edu/courses/7017}}

\noindent Textbook Problems from Strang, \emph{Intro Linear Algebra}, 5th ed.~will be graded for completion.  Answers/solutions to these Problems are linked at

\href{https://bueler.github.io/math314/resources.html}{\texttt{bueler.github.io/math314/resources.html}}

\noindent The \textbf{P} Problems will be graded for correctness.  When grading these Problems, I will expect you to write explanations using complete sentences!
\end{quote}
\medskip

\thispagestyle{empty}

\noindent \hrulefill

\noindent \emph{Put these Textbook Problems first on your PDF, in this order.}

\probset{6.1}{297--302} \# 1, 5, 8, 14, 16, 21, 24, 33

\probset{6.2}{313--317} \# 1, 2, 4, 7, 15, 20

\bigskip
\noindent \hrulefill

\noindent \emph{Put these \textbf{P} Problems next on your PDF, in this order.}

\prob{P52}  \ppart{a}  Compute the eigenvalues and eigenvectors of $\ds A = \begin{bmatrix} 0 & 2 \\ 1 & 1 \end{bmatrix}$.

\epart{b}  Find the eigenvalues and eigenvectors of the matrices the matrices $A+I$ and $A^{-1}$.  Confirm that the eigenvectors are the same as those for $A$.

\epart{c}  Apply the functions $f(x) = x+1$ and $g(x) = 1/x$ to the eigenvalues of $A$, and confirm this gives the same eigenvalues computed in \textbf{(b)}.

\medskip
\noindent \emph{This illustrates a more general rule, that functions of a matrix, like $f(A)=A+I$ and $g(A)=A^{-1}$, have eigenvalues which are computed by the same function, but on complex numbers.}


\prob{P53}  \ppart{a}  Compute the eigenvalues of the matrices
    $$A = \begin{bmatrix} 0 & 0 & 1 \\ 0 & 2 & 0 \\ 3 & 0 & 0 \end{bmatrix}, \qquad B = \begin{bmatrix} 2 & 2 & 2 \\ 2 & 2 & 2 \\ 2 & 2 & 2 \end{bmatrix}.$$

\epart{b}  The matrix $B$ is rank one.  To consider this case, suppose $\bu,\bv$ are nonzero (column) vectors in $\RR^n$.  Let $C = \bu \bv^\top$.  (Recall this is the general formula for a rank one matrix!)  Show that $C \bu = \lambda \bu$ and find $\lambda$.  Also show that if $\bw$ is any vector orthogonal to $\bv$ then $C \bw = \bzero$.  What can you conclude about the eigenvalues of rank one matrices?


\prob{P54}  $2$ by $2$ rotation matrices have the form
    $$A = \begin{bmatrix} \cos \theta & - \sin \theta \\ \sin \theta & \cos \theta \end{bmatrix}$$
Using this form, find a non-identity matrix $A$ with the property that $A^3=I$.  Now compute the eigenvalues of $A$, and confirm that the eigenvalues are certain complex numbers $\lambda$, on the unit circle, which satisfy $\lambda^3=1$.


\prob{P55}  Answer true or false; if true give an explanation and if false give a counterexample.  Assume $A$ is $3$ by $3$.

\epart{a}  If the eigenvalues of $A$ are $2,2,5$ then $A$ is invertible.

\epart{b}  If the eigenvalues of $A$ are $2,2,5$ then $A$ is diagonalizable.


\prob{P56}  The following four diagonalizable $2\times 2$ matrices look rather similar to each other, for instance in terms of the sizes of the entries:
    $$A = \begin{bmatrix} 3 & 2 \\ 1 & 4 \end{bmatrix}, \qquad B = \begin{bmatrix} 3 & 2 \\ -5 & -3 \end{bmatrix}, \qquad C = \begin{bmatrix} 5 & 7 \\ -3 & -4 \end{bmatrix}, \qquad D = \begin{bmatrix} 5 & 6.9 \\ -3 & -4 \end{bmatrix}$$
However, their \emph{powers} act in completely different ways.  Looking at the eigenvalues will explain it.

\epart{a}  Compute the eigenvalues of the 4 matrices and put all 8 values on a single graph in the complex plane.  (\emph{The eigenvalues of real matrices are generally complex!})  Show the unit circle on your graph, and then show the eigenvalues as 8 clearly-labeled dots in the complex plane; consider using a color for each matrix.  Draw the picture well enough so that you can tell where all the eigenvalues are relative to the unit circle.
% see teaching/M314S22/solns/figs/fourmats.m

\epart{b}  Recall that if $M$ is a square and diagonalizable matrix with eigenvalues $\lambda_i$ and eigenvectors $\bx_i$, so that $M=X \Lambda X^{-1}$ is the diagonalization, then
    $$M^{k} = X \Lambda^{k} X^{-1}.$$
(\emph{Remember the calculation:} $M^2 = X \Lambda X^{-1} X \Lambda X^{-1} = X \Lambda^2 X^{-1}$.)  That is, powers of a diagonalized matrix can be computed simply by exponentiating the diagonal matrix of eigenvalues $\Lambda$.  Based on where the eigenvalues in part \textbf{(a)} are, relative to the unit circle, match $A,B,C,D$ to these descriptions:
\renewcommand{\labelenumi}{\arabic{enumi}.}
\begin{enumerate}
\item $M^{100}=I$
\item $M^{100}=-M$
\item $M^{100}$ has tiny entries
\item $M^{100}$ has enormous entries
\end{enumerate}
(\emph{Hint.} A power of a complex number which is on the unit circle just rotates around the circle.  In general, $(r e^{i\theta})^k = r^k e^{ik\theta}$.)

\epart{c}  Use Matlab to compute the 100th power of the 4 matrices to confirm your results.
\end{document}
