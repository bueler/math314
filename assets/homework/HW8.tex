\documentclass[12pt]{amsart}
%prepared in AMSLaTeX, under LaTeX2e
\addtolength{\oddsidemargin}{-.65in}
\addtolength{\evensidemargin}{-.65in}
\addtolength{\topmargin}{-.4in}
\addtolength{\textwidth}{1.4in}
\addtolength{\textheight}{0.8in}

\renewcommand{\baselinestretch}{1.05}

\usepackage{xspace,bm}
\usepackage{verbatim,fancyvrb}
\usepackage{palatino}
\usepackage[final]{graphicx}
\usepackage[pdftex,colorlinks=True]{hyperref}

\newtheorem*{thm}{Theorem}
\newtheorem*{defn}{Definition}
\newtheorem*{example}{Example}
\newtheorem*{problem}{Problem}
\newtheorem*{remark}{Remark}

\newcommand{\mfile}[1]{
\begin{quote}
\bigskip
\VerbatimInput[frame=single,framesep=3mm,label=\fbox{\normalsize \textsl{\,#1\,}},fontfamily=courier,fontsize=\footnotesize]{../matlab/#1}
\medskip
\end{quote}
}

%\DefineVerbatimEnvironment{mVerb}{Verbatim}{numbersep=2mm,frame=lines,framerule=0.1mm,framesep=2mm,xleftmargin=4mm,fontsize=\footnotesize}
\DefineVerbatimEnvironment{mVerb}{Verbatim}{numbersep=2mm,xleftmargin=4mm,fontsize=\footnotesize}

% macros
\usepackage{amssymb}

\newcommand{\ba}{\bm{a}}
\newcommand{\bb}{\bm{b}}
\newcommand{\bc}{\bm{c}}
\newcommand{\br}{\bm{r}}
\newcommand{\bu}{\bm{u}}
\newcommand{\bv}{\bm{v}}
\newcommand{\bw}{\bm{w}}
\newcommand{\bx}{\bm{x}}
\newcommand{\by}{\bm{y}}
\newcommand{\bz}{\bm{z}}

\newcommand{\bR}{\bm{R}}
\newcommand{\bS}{\bm{S}}
\newcommand{\bV}{\bm{V}}

\newcommand{\bzero}{\bm{0}}

\newcommand{\CC}{\mathbb{C}}
\newcommand{\RR}{\mathbb{R}}
\newcommand{\ZZ}{\mathbb{Z}}

\newcommand{\eps}{\epsilon}
\newcommand{\grad}{\nabla}
\newcommand{\lam}{\lambda}
\newcommand{\lap}{\triangle}

\newcommand{\ip}[2]{\ensuremath{\left<#1,#2\right>}}

\newcommand{\image}{\operatorname{im}}
\newcommand{\onull}{\operatorname{null}}
\newcommand{\rank}{\operatorname{rank}}
\newcommand{\range}{\operatorname{range}}
\newcommand{\trace}{\operatorname{tr}}

\newcommand{\ds}{\displaystyle}

\newcommand{\prob}[1]{\bigskip\noindent\textbf{#1.}\quad }
\newcommand{\probset}[2]{\bigskip\noindent\textbf{from Problem Set #1, pages #2:}\quad }

\newcommand{\pts}[1]{(\emph{#1 pts}) }
\newcommand{\epart}[1]{\medskip\noindent\textbf{(#1)}\quad }
\newcommand{\ppart}[1]{\,\textbf{(#1)}\quad }

\newcommand{\Matlab}{\textsc{Matlab}\xspace}
\newcommand{\Octave}{\textsc{Octave}\xspace}


\begin{document}
\scriptsize \noindent Math 314 Linear Algebra (Bueler) \hfill Spring 2022
\normalsize\medskip

\Large
\centerline{\textbf{Homework \#8} (\emph{revised \textbf{P42}})}

\normalsize
\bigskip
\begin{quote}
Due Monday 21 March, 2022 at 11:59pm.

\medskip
\noindent Submit as a single PDF via Gradescope, linked from the Canvas page

\href{https://canvas.alaska.edu/courses/7017}{\texttt{canvas.alaska.edu/courses/7017}}

\noindent Textbook Problems from Strang, \emph{Intro Linear Algebra}, 5th ed.~will be graded for completion.  Answers/solutions to these Problems are linked at

\href{https://bueler.github.io/math314/resources.html}{\texttt{bueler.github.io/math314/resources.html}}

\noindent The \textbf{P} Problems will be graded for correctness.  When grading these Problems, I will expect you to write explanations using complete sentences!
\end{quote}
\medskip

\thispagestyle{empty}

\noindent \hrulefill

\noindent \emph{Put these Textbook Problems first on your PDF, in this order.}

\probset{3.5}{189--192} \# 1, 3, 4, 6, 7, 11, 13, 23

\probset{4.1}{201--204} \# 2, 4, 5, 6, 10, 11, 12, 13, 20


\bigskip
\noindent \hrulefill

\noindent \emph{Put these \textbf{P} Problems next on your PDF, in this order.}

\prob{P36}  I have a $3\times 5$ matrix and I compute its \texttt{rref}:
    $$A = \begin{bmatrix} 0 & 1 & 2 & 3 & 4 \\
                          0 & 2 & 4 & 3 & 2 \\
                          0 & 0 & 0 & 1 & 2 \end{bmatrix}
      \qquad \to \qquad
      R = \begin{bmatrix} 0 & 1 & 2 & 0 & -2 \\
                          0 & 0 & 0 & 1 & 2 \\
                          0 & 0 & 0 & 0 & 0 \end{bmatrix}$$
Find a basis for each of the four subspaces associated to $A$:
    $$\text{row space } C(A^\top), \quad \text{column space } C(A), \quad \text{null space } N(A), \quad \text{left nullspace } N(A^\top)$$

\prob{P37}  Suppose I have factored $A$ as $LU$:
    $$A = \begin{bmatrix} 1 & 2 & 3 & 4 \\
                         1 & 2 & 4 & 6 \\
                         0 & 0 & 1 & 2 \end{bmatrix}
        = \begin{bmatrix} 1 & 0 & 0 \\ 1 & 1 & 0 \\ 0 & 1 & 1 \end{bmatrix}
          \begin{bmatrix} 1 & 2 & 3 & 4 \\ 0 & 0 & 1 & 2 \\ 0 & 0 & 0 & 0 \end{bmatrix}.$$
Find a basis for each of the four subspaces associated to $A$:
    $$\text{row space } C(A^\top), \quad \text{column space } C(A), \quad \text{null space } N(A), \quad \text{left nullspace } N(A^\top)$$


\prob{P38}  Let $\bV$ be the subspace of $\RR^3$ spanned by $(1,-1,0)$ and $(0,2,1)$.

\epart{a} Find a matrix $A$ that has $\bV$ as its row space.

\epart{b} Find a matrix $B$ that has $\bV$ as its null space.

\epart{c} Multiply $AB^\top$.  Multiply $BA^\top$.  Why do these come out so simple?


\prob{P39}  Let $I$ be the $3\times 3$ identity matrix and $O$ be the $3\times 2$ zero matrix.  For each of these matrices, find the dimensions of the four subspaces:
    $$A = \begin{bmatrix} O \end{bmatrix}, \quad B = \begin{bmatrix} I & O \end{bmatrix}, \quad C = \begin{bmatrix} I & I \\ \,O^\top & O^\top \end{bmatrix}$$


\prob{P40} \ppart{a}  Prove that every $\bx$ in $N(A)$ is perpendicular to every $A^\top \by$ in the row space of $A$ (i.e.~in $C(A^\top)$).  \emph{Hint. Start with $A \bx = \bzero$.  Now compute a dot product.}

\epart{b}  Prove that every $\by$ in $N(A^\top)$ is perpendicular to every $A \bx$ in the column space of $A$ (i.e.~in $C(A)$).


\prob{P41}  This system of equations $A\bx = \bb$ has no solutions:
\begin{align*}
2 x + 3 y + 4 z &= 9 \\
4 x + 3 y + 2 z &= 9 \\
2 x \phantom{+sxx} - 2 z &= 1
\end{align*}
Find numbers $c_1,c_2,c_3$ to multiply the equations so that they add to $0=1$.  (\emph{Hint.  Do row operations, and keep track of them.})  You have found a vector $\by$ in which subspace?  Check its dot product: $\by^\top \bb = 1$.


\prob{P42}  Suppose $\bS$ is the subspace spanned by the vectors $(1,2,2,3)$ and $(1,1,1,1)$.  Find two vectors that span the orthogonal complement of $\bS$.  This is the same as solving $A\bx = \bzero$ for which $A$?

\end{document}
