\documentclass[12pt]{amsart}
%prepared in AMSLaTeX, under LaTeX2e
\addtolength{\oddsidemargin}{-.65in} 
\addtolength{\evensidemargin}{-.65in}
\addtolength{\topmargin}{-.4in}
\addtolength{\textwidth}{1.4in}
\addtolength{\textheight}{0.8in}

\renewcommand{\baselinestretch}{1.05}

\usepackage{xspace,bm}
\usepackage{verbatim,fancyvrb}
\usepackage{palatino}
\usepackage[final]{graphicx}
\usepackage[pdftex,colorlinks=True]{hyperref}

\newtheorem*{thm}{Theorem}
\newtheorem*{defn}{Definition}
\newtheorem*{example}{Example}
\newtheorem*{problem}{Problem}
\newtheorem*{remark}{Remark}

\newcommand{\mfile}[1]{
\begin{quote}
\bigskip
\VerbatimInput[frame=single,framesep=3mm,label=\fbox{\normalsize \textsl{\,#1\,}},fontfamily=courier,fontsize=\footnotesize]{../matlab/#1}
\medskip
\end{quote}
}

%\DefineVerbatimEnvironment{mVerb}{Verbatim}{numbersep=2mm,frame=lines,framerule=0.1mm,framesep=2mm,xleftmargin=4mm,fontsize=\footnotesize}
\DefineVerbatimEnvironment{mVerb}{Verbatim}{numbersep=2mm,xleftmargin=4mm,fontsize=\footnotesize}

% macros
\usepackage{amssymb}

\newcommand{\ba}{\bm{a}}
\newcommand{\bb}{\bm{b}}
\newcommand{\bc}{\bm{c}}
\newcommand{\br}{\bm{r}}
\newcommand{\bu}{\bm{u}}
\newcommand{\bv}{\bm{v}}
\newcommand{\bw}{\bm{w}}
\newcommand{\bx}{\bm{x}}
\newcommand{\by}{\bm{y}}
\newcommand{\bz}{\bm{z}}

\newcommand{\CC}{\mathbb{C}}
\newcommand{\RR}{\mathbb{R}}
\newcommand{\ZZ}{\mathbb{Z}}

\newcommand{\eps}{\epsilon}
\newcommand{\grad}{\nabla}
\newcommand{\lam}{\lambda}
\newcommand{\lap}{\triangle}

\newcommand{\ip}[2]{\ensuremath{\left<#1,#2\right>}}

\newcommand{\image}{\operatorname{im}}
\newcommand{\onull}{\operatorname{null}}
\newcommand{\rank}{\operatorname{rank}}
\newcommand{\range}{\operatorname{range}}
\newcommand{\trace}{\operatorname{tr}}

\newcommand{\ds}{\displaystyle}

\newcommand{\prob}[1]{\bigskip\noindent\textbf{#1}\quad }

\newcommand{\probset}[2]{\prob{from Problem Set #1, pages #2:}}

\newcommand{\pts}[1]{(\emph{#1 pts}) }
\newcommand{\epart}[1]{\medskip\noindent\textbf{(#1)}\quad }
\newcommand{\ppart}[1]{\,\textbf{(#1)}\quad }

\newcommand{\Matlab}{\textsc{Matlab}\xspace}
\newcommand{\Octave}{\textsc{Octave}\xspace}


\begin{document}
\scriptsize \noindent Math 314 Linear Algebra (Bueler) \hfill Spring 2022
\normalsize\medskip

\Large
\centerline{\textbf{Homework \#4}}

\normalsize
\bigskip
\begin{quote}
Due Monday 7 February, 2022 at 11:59pm.

\medskip
\noindent Submit as a single PDF via Gradescope, linked from the Canvas page

\href{https://canvas.alaska.edu/courses/7017}{\texttt{canvas.alaska.edu/courses/7017}}

\noindent Textbook Problems from Strang, \emph{Intro Linear Algebra}, 5th ed.~will be graded for completion.  Answers/solutions are linked at

\href{https://bueler.github.io/math314/resources.html}{\texttt{bueler.github.io/math314/resources.html}}

\noindent \textbf{P} Problems will be graded for correctness.
\end{quote}
\medskip

\thispagestyle{empty}

\noindent \hrulefill

\noindent \emph{Put these Textbook Problems first on your PDF, in this order.}

\probset{2.4}{77--82} \# 1, 2, 5, 15, 23, 32

\probset{2.5}{92--96} \# 6, 11, 12, 16, 18, 25, 29, 34


\bigskip
\noindent \hrulefill

\noindent \emph{Put these \textbf{P} Problems next on your PDF, in this order.}

\prob{P17.}  Assume $A$ and $B$ are square matrices of the same size.  Which of the following matrices are guaranteed to equal $(A+B)^2$?
    $$A^2 + B^2, \quad A^2 + 2 A B + B^2, \quad A(A+B) + B(A+B), \quad A^2 + AB + BA + B^2$$
Explain why if so, and provide a counter-example if not.  (\emph{Hint: When not equal, either $1 \times 1$ or $2\times 2$ counterexamples will suffice.})

\prob{P18.}  If $A$ is $m \times n$, how many multiplications are needed when

\epart{a} $A$ multiplies a column vector $\bx$ of size $n$?

\epart{b} $A$ multiplies an $n\times k$ matrix $B$?

\epart{c} $A$ multiplies itself to produce $A^2$, in the case where $m=n$?

\prob{P19.}  Use the Gauss-Jordan method to calculate $A^{-1}$ when
    $$A = \begin{bmatrix} 2 & 1 & 0 \\ 1 & 2 & 1 \\ 0 & 1 & 2 \end{bmatrix}$$
(\emph{That is, eliminate above and below the pivots as you convert $\big[A\,\,\, I\big]$ to $\big[I\,\,\, A^{-1}\big]$}, and show your steps.)  Check your result using Matlab's \texttt{inv()} command.

\prob{P20.}  There are sixteen $2\times 2$ matrices whose entries are 0's and 1's only.  How many of the 16 are invertible?

\end{document}
