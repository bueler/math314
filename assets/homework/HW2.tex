\documentclass[12pt]{amsart}
%prepared in AMSLaTeX, under LaTeX2e
\addtolength{\oddsidemargin}{-.65in} 
\addtolength{\evensidemargin}{-.65in}
\addtolength{\topmargin}{-.4in}
\addtolength{\textwidth}{1.4in}
\addtolength{\textheight}{0.8in}

\renewcommand{\baselinestretch}{1.05}

\usepackage{xspace,bm}
\usepackage{verbatim,fancyvrb}
\usepackage{palatino}
\usepackage[final]{graphicx}
\usepackage[pdftex,colorlinks=True]{hyperref}

\newtheorem*{thm}{Theorem}
\newtheorem*{defn}{Definition}
\newtheorem*{example}{Example}
\newtheorem*{problem}{Problem}
\newtheorem*{remark}{Remark}

\newcommand{\mfile}[1]{
\begin{quote}
\bigskip
\VerbatimInput[frame=single,framesep=3mm,label=\fbox{\normalsize \textsl{\,#1\,}},fontfamily=courier,fontsize=\footnotesize]{../matlab/#1}
\medskip
\end{quote}
}

%\DefineVerbatimEnvironment{mVerb}{Verbatim}{numbersep=2mm,frame=lines,framerule=0.1mm,framesep=2mm,xleftmargin=4mm,fontsize=\footnotesize}
\DefineVerbatimEnvironment{mVerb}{Verbatim}{numbersep=2mm,xleftmargin=4mm,fontsize=\footnotesize}

% macros
\usepackage{amssymb}

\newcommand{\ba}{\bm{a}}
\newcommand{\bb}{\bm{b}}
\newcommand{\bc}{\bm{c}}
\newcommand{\br}{\bm{r}}
\newcommand{\bu}{\bm{u}}
\newcommand{\bv}{\bm{v}}
\newcommand{\bw}{\bm{w}}
\newcommand{\bx}{\bm{x}}
\newcommand{\by}{\bm{y}}
\newcommand{\bz}{\bm{z}}

\newcommand{\CC}{\mathbb{C}}
\newcommand{\RR}{\mathbb{R}}
\newcommand{\ZZ}{\mathbb{Z}}

\newcommand{\eps}{\epsilon}
\newcommand{\grad}{\nabla}
\newcommand{\lam}{\lambda}
\newcommand{\lap}{\triangle}

\newcommand{\ip}[2]{\ensuremath{\left<#1,#2\right>}}

\newcommand{\image}{\operatorname{im}}
\newcommand{\onull}{\operatorname{null}}
\newcommand{\rank}{\operatorname{rank}}
\newcommand{\range}{\operatorname{range}}
\newcommand{\trace}{\operatorname{tr}}

\newcommand{\ds}{\displaystyle}

\newcommand{\prob}[1]{\bigskip\noindent\textbf{#1}\quad }

\newcommand{\probset}[2]{\prob{from Problem Set #1, pages #2:}}

\newcommand{\pts}[1]{(\emph{#1 pts}) }
\newcommand{\epart}[1]{\medskip\noindent\textbf{(#1)}\quad }
\newcommand{\ppart}[1]{\,\textbf{(#1)}\quad }

\newcommand{\Matlab}{\textsc{Matlab}\xspace}
\newcommand{\Octave}{\textsc{Octave}\xspace}


\begin{document}
\scriptsize \noindent Math 314 Linear Algebra (Bueler) \hfill Spring 2022
\normalsize\medskip

\Large
\centerline{\textbf{Homework \#2}}

\normalsize
\bigskip
\begin{quote}
Due Monday 24 January, 2022 at 11:59pm.

\medskip
\noindent Submit as a single PDF by using Gradescope, via the course Canvas site

\href{https://canvas.alaska.edu/courses/7017}{\texttt{canvas.alaska.edu/courses/7017}}

\noindent Problems from the textbook (Strang, \emph{Intro Linear Algebra}, 5th ed.~2016) will be graded for completion, while the ``\textbf{P}'' Problems will be graded for correctness.  Answers/solutions to Textbook Problems are linked at

\href{https://bueler.github.io/math314/resources.html}{\texttt{bueler.github.io/math314/resources.html}}
\end{quote}
\medskip

\thispagestyle{empty}

\noindent \hrulefill

\noindent \emph{Put these Textbook Problems first on your PDF, in this order.}

\probset{1.3}{29--30} \# 1, 3, 8, 14

\probset{2.1}{41--45} \# 1, 4, 8, 13, 17, 18, 22, 23, 26, 34


\bigskip
\noindent \hrulefill

\noindent \emph{Put these ``P'' Problems next on your PDF, in this order.}

\prob{P7.}  \ppart{a} Solve this equation $S \by = \bb$ for $\by$.  Note $S$ is a \emph{sum matrix}.
    $$\begin{bmatrix} 1 & 0 & 0 & 0 \\ 1 & 1 & 0 & 0 \\ 1 & 1 & 1 & 0 \\ 1 & 1 & 1 & 1 \end{bmatrix} \begin{bmatrix} y_1 \\ y_2 \\ y_3 \\ y_4 \end{bmatrix} = \begin{bmatrix} 3 \\ 4 \\ 8 \\ 9 \end{bmatrix}.$$

\epart{b} Solve this equation $M \by = \bb$ for $\by$.  Note $M$ is a \emph{difference matrix}.
    $$\begin{bmatrix} 1 & 0 & 0 & 0 \\ -1 & 1 & 0 & 0 \\ 0 & -1 & 1 & 0 \\ 0 & 0 & -1 & 1 \end{bmatrix} \begin{bmatrix} y_1 \\ y_2 \\ y_3 \\ y_4 \end{bmatrix} = \begin{bmatrix} 3 \\ 1 \\ 4 \\ 1 \end{bmatrix}.$$

\epart{c} If I take any vector $\bu$ and first multiply it by $M$ from part \textbf{(b)} to get $M\bu=\bv$, and then I multiply $\bv$ by $S$ from part \textbf{(a)} to get $S\bv=\bw$, what is $\bw$?

\prob{P8.}  Here are three vectors:
    $$\bv_1 = \begin{bmatrix} 2 \\ 3 \\ 4 \end{bmatrix}, \qquad \bv_2 = \begin{bmatrix} -1 \\ 0 \\ 1 \end{bmatrix}, \qquad \bv_3 = \begin{bmatrix} 5 \\ 6 \\ 7 \end{bmatrix}.$$
One may create the zero vector from the linear combination $x_1 \bv_1 + x_2 \bv_2 + x_3 \bv_3$ by choosing $x_1=x_2=x_3=0$, but that is obvious and boring.  Instead, choose $x_1=1$ and find $x_2$ and $x_3$ so that the linear combination is again the zero vector.  Does this show that the three vectors are independent or dependent?  The three vectors lie in a \underline{\phantom{kdafjlas afds}}.  (Note that the matrix $V$ formed from these vectors is \emph{not invertible}.)

\prob{P9.}  \ppart{a} Compute this matrix-vector product by using dot products of the rows with the column vector:
    $$\begin{bmatrix} 3 & -1 & 0 & 0 \\ -1 & 3 & -1 & 0 \\ 0 & -1 & 3 & -1 \\ 0 & 0 & -1 & 3 \end{bmatrix} \begin{bmatrix} 1 \\ 3 \\ 1 \\ 2 \end{bmatrix}.$$

\epart{b}  Compute the same matrix-vector product by a linear combination of the columns of the matrix.

\prob{P10.}  \ppart{a} What $2$ by $2$ matrix $R$ rotates every vector counter-clockwise by $90^\circ$?  (Note $R$ times $\begin{bmatrix} x \\ y \end{bmatrix}$ is $\begin{bmatrix} y \\ -x \end{bmatrix}$.)

\epart{b} What $2$ by $2$ matrix $S$ rotates every vector by $180^\circ$?

\epart{c} Show that for any vector $\bu$, $R(R\bu) = S\bu$.
\end{document}
