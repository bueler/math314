\documentclass[12pt]{amsart}
%prepared in AMSLaTeX, under LaTeX2e
\addtolength{\oddsidemargin}{-.65in} 
\addtolength{\evensidemargin}{-.65in}
\addtolength{\topmargin}{-.4in}
\addtolength{\textwidth}{1.4in}
\addtolength{\textheight}{0.8in}

\renewcommand{\baselinestretch}{1.05}

\usepackage{xspace,bm}
\usepackage{verbatim,fancyvrb}
\usepackage{palatino}
\usepackage[final]{graphicx}
\usepackage[pdftex,colorlinks=True]{hyperref}

\newtheorem*{thm}{Theorem}
\newtheorem*{defn}{Definition}
\newtheorem*{example}{Example}
\newtheorem*{problem}{Problem}
\newtheorem*{remark}{Remark}

\newcommand{\mfile}[1]{
\begin{quote}
\bigskip
\VerbatimInput[frame=single,framesep=3mm,label=\fbox{\normalsize \textsl{\,#1\,}},fontfamily=courier,fontsize=\footnotesize]{../matlab/#1}
\medskip
\end{quote}
}

%\DefineVerbatimEnvironment{mVerb}{Verbatim}{numbersep=2mm,frame=lines,framerule=0.1mm,framesep=2mm,xleftmargin=4mm,fontsize=\footnotesize}
\DefineVerbatimEnvironment{mVerb}{Verbatim}{numbersep=2mm,xleftmargin=4mm,fontsize=\footnotesize}

% macros
\usepackage{amssymb}

\newcommand{\ba}{\bm{a}}
\newcommand{\bb}{\bm{b}}
\newcommand{\bc}{\bm{c}}
\newcommand{\br}{\bm{r}}
\newcommand{\bu}{\bm{u}}
\newcommand{\bv}{\bm{v}}
\newcommand{\bw}{\bm{w}}
\newcommand{\bx}{\bm{x}}
\newcommand{\by}{\bm{y}}
\newcommand{\bz}{\bm{z}}

\newcommand{\bR}{\bm{R}}

\newcommand{\CC}{\mathbb{C}}
\newcommand{\RR}{\mathbb{R}}
\newcommand{\ZZ}{\mathbb{Z}}

\newcommand{\eps}{\epsilon}
\newcommand{\grad}{\nabla}
\newcommand{\lam}{\lambda}
\newcommand{\lap}{\triangle}

\newcommand{\ip}[2]{\ensuremath{\left<#1,#2\right>}}

\newcommand{\image}{\operatorname{im}}
\newcommand{\onull}{\operatorname{null}}
\newcommand{\rank}{\operatorname{rank}}
\newcommand{\range}{\operatorname{range}}
\newcommand{\trace}{\operatorname{tr}}

\newcommand{\ds}{\displaystyle}

\newcommand{\prob}[1]{\bigskip\noindent\textbf{#1.}\quad }

\newcommand{\probset}[2]{\prob{from Problem Set #1, pages #2:}}

\newcommand{\pts}[1]{(\emph{#1 pts}) }
\newcommand{\epart}[1]{\medskip\noindent\textbf{(#1)}\quad }
\newcommand{\ppart}[1]{\,\textbf{(#1)}\quad }

\newcommand{\Matlab}{\textsc{Matlab}\xspace}
\newcommand{\Octave}{\textsc{Octave}\xspace}


\begin{document}
\scriptsize \noindent Math 314 Linear Algebra (Bueler) \hfill Spring 2022
\normalsize\medskip

\Large
\centerline{\textbf{Homework \#6}}

\normalsize
\bigskip
\begin{quote}
Due Monday 21 February, 2022 at 11:59pm.

\medskip
\noindent Submit as a single PDF via Gradescope, linked from the Canvas page

\href{https://canvas.alaska.edu/courses/7017}{\texttt{canvas.alaska.edu/courses/7017}}

\noindent Textbook Problems from Strang, \emph{Intro Linear Algebra}, 5th ed.~will be graded for completion.  Answers/solutions are linked at

\href{https://bueler.github.io/math314/resources.html}{\texttt{bueler.github.io/math314/resources.html}}

\noindent \textbf{P} Problems will be graded for correctness.  When grading these, I will expect you to write explanations using complete sentences!
\end{quote}
\medskip

\thispagestyle{empty}

\noindent \hrulefill

\noindent \emph{Put these Textbook Problems first on your PDF, in this order.}

\probset{3.1}{130--133} \# 4, 5, 12, 14, 17, 20, 23, 27

\probset{3.2}{141--147} \# 1, 2, 7, 10, 15, 17, 22


\bigskip
\noindent \hrulefill

\noindent \emph{Put these \textbf{P} Problems next on your PDF, in this order.}

\prob{P21}  If no zero pivots appear along the way, elimination can factor a symmetric matrix $S$ into $S=L D L^\top$ where $L$ is lower triangular, with ones on the diagonal as usual, and $D$ is a diagonal matrix.  The calculation proceeds essentially the same as an LU factorization, but once we see $U$ we can, because of symmetry, ``pull out'' the diagonal entries from $U$, and the remaining upper triangular matrix will be the transpose of the $L$ factor.  Specifically, the diagonal matrix $D$ is formed from the pivots.

Do this factorization on the following matrices:
    $$S = \begin{bmatrix} 1 & 3 \\ 3 & 2 \end{bmatrix}, \quad S = \begin{bmatrix} 2 & -1 & 0 & 0 \\ -1 & 2 & -1 & 0 \\ 0 & -1 & 2 & -1 \\ 0 & 0 & -1 & 2 \end{bmatrix}$$
Check the factorization $S=L D L^\top$ by multiplying back!
\begin{comment}
>> S = [2 -1 0 0; -1 2 -1 0; 0 -1 2 -1; 0 0 -1 2];
>> Z = S;
>> Z(2,:) = Z(2,:) - (-1/2) * Z(1,:);
>> Z(3,:) = Z(3,:) - (-1/(3/2)) * Z(2,:);
>> Z(4,:) = Z(4,:) - (-1/(4/3)) * Z(3,:);
>> U = Z
U =
   2.00000  -1.00000   0.00000   0.00000
   0.00000   1.50000  -1.00000   0.00000
   0.00000   0.00000   1.33333  -1.00000
   0.00000   0.00000   0.00000   1.25000
>> L = eye(4); L(2,1) = -1/2; L(3,2) = -2/3; L(4,3) = -3/4
L =
   1.00000   0.00000   0.00000   0.00000
  -0.50000   1.00000   0.00000   0.00000
   0.00000  -0.66667   1.00000   0.00000
   0.00000   0.00000  -0.75000   1.00000
>> D = diag(diag(U))
D =
   2.00000   0.00000   0.00000   0.00000
   0.00000   1.50000   0.00000   0.00000
   0.00000   0.00000   1.33333   0.00000
   0.00000   0.00000   0.00000   1.25000
>> norm(S - L*D*L')
ans = 0
\end{comment}


\prob{P22}  \ppart{a}  How many entries of a symmetric $5\times 5$ matrix $S$ can be chosen independently?

\epart{b}  Large, symmetric matrices, of size $m\times m$, require about half the storage (memory) on a computer as general matrices of the same size if the entries are saved in a careful scheme.  Explain.

\epart{c}  A \emph{skew-symmetric} matrix $A$ is one for which $A^\top = -A$.  How many entries of a skew-symmetric $5\times 5$ matrix $A$ can be chosen independently?


\prob{P23} \ppart{a}  Assume $A$ is $m\times n$.  Explain why the product $A^\top A$ is always defined; what size is it?  Then explain why
    $$(A^\top A)_{ij} = \sum_{k=1}^m a_{ki}\, a_{kj}$$
    % = \sum_{k=1}^m (A^\top)_{ik} a_{kj}
(\emph{Hint.  Start from the general formula for $(AB)_{ij}$, but then specialize to the current case.})

\epart{b}  Show that if $A$ is not a zero matrix then $A^\top A$ is also not a zero matrix.

\epart{c}  Find a nonzero matrix $A$ so that $A^2=0$.  Then calculate $A^\top A$ and confirm it is \emph{not} zero.


\prob{P24}  Which of the following subsets of $\bR^3$ are actually subspaces?  Explain.

\epart{a}  The plane of vectors $(b_1,b_2,b_3)$ with $b_1=b_3$.

\epart{b}  The plane of vectors with $b_1=1$.

\epart{c}  The vectors with $b_1 b_2 b_3 = 0$.

\epart{d}  All linear combinations of $\bv = (3,1,0)$ and $\bw = (2,2,2)$.

\epart{e}  All vectors which satisfy $b_1+b_2+b_3=0$.

\epart{f}  All vectors with $b_1 \ge b_2 \ge b_3$.


\prob{P25}  In each part, describe the smallest subspace of the $2\times 2$ matrix space $M$ which contains the given matrices.  (\emph{Hint.  Answer by giving a parameterized general form for a matrix in the subspace.})

\epart{a} $\begin{bmatrix} 1 & 0 \\ 0 & 0 \end{bmatrix}$ and $\begin{bmatrix} 0 & 1 \\ 0 & 1 \end{bmatrix}$

\epart{b} $\begin{bmatrix} 1 & 0 \\ 0 & 0 \end{bmatrix}$ and $\begin{bmatrix} 1 & 0 \\ 0 & 1 \end{bmatrix}$


\prob{P26}  Is it possible to construct a matrix $A$ whose column space contains $(1,1,0)$ and $(0,1,1)$, and whose nullspace contains $(1,0,1)$ and $(0,0,1)$.  Explain your answer.  (\emph{Hint.  What size is $A$?  How many pivots and how many free variables?})


\prob{P27}  If $A$ is $4\times 4$ and invertible, describe the nullspace of the $4\times 8$ matrix $B=[A\,\,A]$.

\end{document}
