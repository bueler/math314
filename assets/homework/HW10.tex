\documentclass[12pt]{amsart}
%prepared in AMSLaTeX, under LaTeX2e
\addtolength{\oddsidemargin}{-.65in}
\addtolength{\evensidemargin}{-.65in}
\addtolength{\topmargin}{-.4in}
\addtolength{\textwidth}{1.4in}
\addtolength{\textheight}{0.8in}

\renewcommand{\baselinestretch}{1.05}

\usepackage{xspace,bm}
\usepackage{verbatim,fancyvrb}
\usepackage{palatino}
\usepackage[final]{graphicx}
\usepackage[pdftex,colorlinks=True]{hyperref}

\newtheorem*{thm}{Theorem}
\newtheorem*{defn}{Definition}
\newtheorem*{example}{Example}
\newtheorem*{problem}{Problem}
\newtheorem*{remark}{Remark}

\newcommand{\mfile}[1]{
\begin{quote}
\bigskip
\VerbatimInput[frame=single,framesep=3mm,label=\fbox{\normalsize \textsl{\,#1\,}},fontfamily=courier,fontsize=\footnotesize]{../matlab/#1}
\medskip
\end{quote}
}

%\DefineVerbatimEnvironment{mVerb}{Verbatim}{numbersep=2mm,frame=lines,framerule=0.1mm,framesep=2mm,xleftmargin=4mm,fontsize=\footnotesize}
\DefineVerbatimEnvironment{mVerb}{Verbatim}{numbersep=2mm,xleftmargin=4mm,fontsize=\footnotesize}

% macros
\usepackage{amssymb}

\newcommand{\ba}{\bm{a}}
\newcommand{\bb}{\bm{b}}
\newcommand{\bc}{\bm{c}}
\newcommand{\bp}{\bm{p}}
\newcommand{\br}{\bm{r}}
\newcommand{\bu}{\bm{u}}
\newcommand{\bv}{\bm{v}}
\newcommand{\bw}{\bm{w}}
\newcommand{\bx}{\bm{x}}
\newcommand{\by}{\bm{y}}
\newcommand{\bz}{\bm{z}}

\newcommand{\bR}{\bm{R}}
\newcommand{\bS}{\bm{S}}
\newcommand{\bV}{\bm{V}}

\newcommand{\bzero}{\bm{0}}

\newcommand{\CC}{\mathbb{C}}
\newcommand{\RR}{\mathbb{R}}
\newcommand{\ZZ}{\mathbb{Z}}

\newcommand{\eps}{\epsilon}
\newcommand{\grad}{\nabla}
\newcommand{\lam}{\lambda}
\newcommand{\lap}{\triangle}

\newcommand{\ip}[2]{\ensuremath{\left<#1,#2\right>}}

\newcommand{\image}{\operatorname{im}}
\newcommand{\onull}{\operatorname{null}}
\newcommand{\rank}{\operatorname{rank}}
\newcommand{\range}{\operatorname{range}}
\newcommand{\trace}{\operatorname{tr}}

\newcommand{\ds}{\displaystyle}

\newcommand{\prob}[1]{\bigskip\noindent\textbf{#1.}\quad }
\newcommand{\probset}[2]{\bigskip\noindent\textbf{from Problem Set #1, pages #2:}\quad }

\newcommand{\pts}[1]{(\emph{#1 pts}) }
\newcommand{\epart}[1]{\medskip\noindent\textbf{(#1)}\quad }
\newcommand{\ppart}[1]{\,\textbf{(#1)}\quad }

\newcommand{\Matlab}{\textsc{Matlab}\xspace}
\newcommand{\Octave}{\textsc{Octave}\xspace}


\begin{document}
\scriptsize \noindent Math 314 Linear Algebra (Bueler) \hfill Spring 2022
\normalsize\medskip

\Large
\centerline{\textbf{Homework \#10}}

\bigskip
\large
\centerline{Due Monday 4 April, 2022 at 11:59pm.}

\normalsize
\bigskip
\begin{quote}
\medskip
\noindent Submit as a single PDF via Gradescope; see the Canvas page

\href{https://canvas.alaska.edu/courses/7017}{\texttt{canvas.alaska.edu/courses/7017}}

\noindent Textbook Problems from Strang, \emph{Intro Linear Algebra}, 5th ed.~will be graded for completion.  Answers/solutions to these Problems are linked at

\href{https://bueler.github.io/math314/resources.html}{\texttt{bueler.github.io/math314/resources.html}}

\noindent The \textbf{P} Problems will be graded for correctness.  When grading these Problems, I will expect you to write explanations using complete sentences!
\end{quote}
\medskip

\thispagestyle{empty}

\noindent \hrulefill

\noindent \emph{Put these Textbook Problems first on your PDF, in this order.}

\probset{5.1}{253--256} \# 1, 3, 4, 7, 10, 18, 19

\probset{5.2}{265--271} \# x


\bigskip
\noindent \hrulefill

\noindent \emph{Put these \textbf{P} Problems next on your PDF, in this order.}

\prob{P47} \emph{This problem refers to ideas in section 4.4.}

Suppose
  $$Z = \begin{bmatrix} 1/3 & -1/3 & 1/3 \\ -1/3 & 5/6 & 1/6 \\ 1/3 & 1/6 & 5/6 \end{bmatrix}$$ 

\epart{a}  Check that the columns of this matrix $Z$ form an orthonormal basis.  That is, check that the dot products of distinct columns give zero, while the length of each column is one.

\epart{b}  Check that $Z^\top Z = I$.

\epart{c}  In terms of how we understand matrix multiplication, explain briefly why calculations \textbf{(a)} and \textbf{(b)} are the same.  Why is it significant that $Z^\top Z$ is symmetric?

\prob{P48} like 5.1 \#13/14

\prob{P49} 5.1 \#28

\prob{P50} 5.1 \#29

\prob{P51} 5.1 \#32

\prob{P52} det(orthogonal) = +-1: example, show generally

\end{document}
